%% Generated by Sphinx.
\def\sphinxdocclass{report}
\documentclass[letterpaper,10pt,english]{sphinxmanual}
\ifdefined\pdfpxdimen
   \let\sphinxpxdimen\pdfpxdimen\else\newdimen\sphinxpxdimen
\fi \sphinxpxdimen=.75bp\relax
\ifdefined\pdfimageresolution
    \pdfimageresolution= \numexpr \dimexpr1in\relax/\sphinxpxdimen\relax
\fi
%% let collapsible pdf bookmarks panel have high depth per default
\PassOptionsToPackage{bookmarksdepth=5}{hyperref}

\PassOptionsToPackage{warn}{textcomp}
\usepackage[utf8]{inputenc}
\ifdefined\DeclareUnicodeCharacter
% support both utf8 and utf8x syntaxes
  \ifdefined\DeclareUnicodeCharacterAsOptional
    \def\sphinxDUC#1{\DeclareUnicodeCharacter{"#1}}
  \else
    \let\sphinxDUC\DeclareUnicodeCharacter
  \fi
  \sphinxDUC{00A0}{\nobreakspace}
  \sphinxDUC{2500}{\sphinxunichar{2500}}
  \sphinxDUC{2502}{\sphinxunichar{2502}}
  \sphinxDUC{2514}{\sphinxunichar{2514}}
  \sphinxDUC{251C}{\sphinxunichar{251C}}
  \sphinxDUC{2572}{\textbackslash}
\fi
\usepackage{cmap}
\usepackage[T1]{fontenc}
\usepackage{amsmath,amssymb,amstext}
\usepackage{babel}



\usepackage{tgtermes}
\usepackage{tgheros}
\renewcommand{\ttdefault}{txtt}



\usepackage[Bjarne]{fncychap}
\usepackage{sphinx}

\fvset{fontsize=auto}
\usepackage{geometry}


% Include hyperref last.
\usepackage{hyperref}
% Fix anchor placement for figures with captions.
\usepackage{hypcap}% it must be loaded after hyperref.
% Set up styles of URL: it should be placed after hyperref.
\urlstyle{same}

\addto\captionsenglish{\renewcommand{\contentsname}{Contents:}}

\usepackage{sphinxmessages}
\setcounter{tocdepth}{1}

\usepackage{enumitem}
\setlistdepth{99}

\title{pywfe}
\date{Nov 10, 2023}
\release{}
\author{Austen Stone}
\newcommand{\sphinxlogo}{\vbox{}}
\renewcommand{\releasename}{}
\makeindex
\begin{document}

\ifdefined\shorthandoff
  \ifnum\catcode`\=\string=\active\shorthandoff{=}\fi
  \ifnum\catcode`\"=\active\shorthandoff{"}\fi
\fi

\pagestyle{empty}
\sphinxmaketitle
\pagestyle{plain}
\sphinxtableofcontents
\pagestyle{normal}
\phantomsection\label{\detokenize{index::doc}}



\chapter{pywfe \sphinxhyphen{} a python package for the wave finite element method}
\label{\detokenize{index:pywfe-a-python-package-for-the-wave-finite-element-method}}
\noindent{\hspace*{\fill}\sphinxincludegraphics[width=1000\sphinxpxdimen]{{logo}.png}\hspace*{\fill}}

\sphinxAtStartPar
This package implements the Wave Finite Element Method (WFEM) in Python to analyse guided waves in 1 dimension. Initially Written to analyse mechanical waves in fluid filled pipes meshed in COMSOL.

\sphinxAtStartPar
\sphinxstylestrong{Currently only works for infinite waveguides.}

\sphinxAtStartPar
The \sphinxtitleref{pywfe.Model} class provides a high level api to calculate the free and forced response in the waveguide.

\sphinxstepscope


\section{pyWFE Package}
\label{\detokenize{pywfe:module-pywfe}}\label{\detokenize{pywfe:pywfe-package}}\label{\detokenize{pywfe::doc}}\index{module@\spxentry{module}!pywfe@\spxentry{pywfe}}\index{pywfe@\spxentry{pywfe}!module@\spxentry{module}}
\sphinxAtStartPar
pywfe. A Python implementation of the WFE method

\sphinxstepscope


\section{pywfe.Model Class}
\label{\detokenize{model:pywfe-model-class}}\label{\detokenize{model::doc}}

\subsection{Introduction}
\label{\detokenize{model:introduction}}
\sphinxAtStartPar
Brief description of the class and its purpose.


\subsection{Constructor}
\label{\detokenize{model:constructor}}\index{\_\_init\_\_() (pywfe.model.Model method)@\spxentry{\_\_init\_\_()}\spxextra{pywfe.model.Model method}}

\begin{fulllineitems}
\phantomsection\label{\detokenize{model:pywfe.model.Model.__init__}}
\pysigstartsignatures
\pysiglinewithargsret{\sphinxcode{\sphinxupquote{Model.}}\sphinxbfcode{\sphinxupquote{\_\_init\_\_}}}{\emph{\DUrole{n}{K}}, \emph{\DUrole{n}{M}}, \emph{\DUrole{n}{dof}}, \emph{\DUrole{n}{null}\DUrole{o}{=}\DUrole{default_value}{None}}, \emph{\DUrole{n}{nullf}\DUrole{o}{=}\DUrole{default_value}{None}}, \emph{\DUrole{n}{axis}\DUrole{o}{=}\DUrole{default_value}{0}}, \emph{\DUrole{n}{logging\_level}\DUrole{o}{=}\DUrole{default_value}{20}}, \emph{\DUrole{n}{solver}\DUrole{o}{=}\DUrole{default_value}{\textquotesingle{}transfer\_matrix\textquotesingle{}}}}{}
\pysigstopsignatures
\sphinxAtStartPar
Initialise a Model object.
\begin{quote}\begin{description}
\sphinxlineitem{Parameters}\begin{description}
\sphinxlineitem{\sphinxstylestrong{K}}{[}np.ndarray{]}
\sphinxAtStartPar
Stiffness matrix \(\mathbf{K}\) of shape \((N, N)\).

\sphinxlineitem{\sphinxstylestrong{M}}{[}np.ndarray{]}
\sphinxAtStartPar
Mass matrix \(\mathbf{M}\) of shape \((N, N)\).

\sphinxlineitem{\sphinxstylestrong{dof}}{[}dict{]}
\sphinxAtStartPar
A dictionary containing the following keys:
\begin{itemize}
\item {} 
\sphinxAtStartPar
‘coord’ : array\sphinxhyphen{}like, shape \((n_{dim}, N)\)
Coordinates of the degrees of freedom, where \(n_{dim}\) is the number of spatial dimensions and \(N\) is the total number of degrees of freedom in the initial total mesh.

\item {} 
\sphinxAtStartPar
‘node’ : array\sphinxhyphen{}like, shape \((N,)\)
Node number that the degree of freedom sits on.

\item {} 
\sphinxAtStartPar
‘fieldvar’ : array\sphinxhyphen{}like, shape \((N,)\)
Field variable for the degree of freedom (e.g., pressure, displacement in x, displacement in y).

\item {} 
\sphinxAtStartPar
‘index’ : array\sphinxhyphen{}like, shape \((N,)\)
Index of the degree of freedom, used to keep track of the degrees of freedom when sorted.

\end{itemize}

\sphinxlineitem{\sphinxstylestrong{null}}{[}ndarray, optional{]}
\sphinxAtStartPar
Null space constraint matrix (for boundary conditions) of shape \((N, N)\). The default is None.

\sphinxlineitem{\sphinxstylestrong{nullf}}{[}ndarray, optional{]}
\sphinxAtStartPar
Force null space constraint matrix (for boundary conditions) of shape \((N, N)\). The default is None.

\sphinxlineitem{\sphinxstylestrong{axis}}{[}int, optional{]}
\sphinxAtStartPar
The waveguide axis. Moves \sphinxtitleref{dof{[}‘coord’{]}{[}axis{]}} to \sphinxtitleref{dof{[}‘coord’{]}{[}0{]}}. The default is 0.

\sphinxlineitem{\sphinxstylestrong{logging\_level}}{[}int, optional{]}
\sphinxAtStartPar
Logging level. The default is 20.

\sphinxlineitem{\sphinxstylestrong{solver}}{[}str, optional{]}
\sphinxAtStartPar
The form of the eigenvalue to use. The default is “transfer\_matrix”.
Options are currently ‘transfer\_matrix’ or ‘polynomial’.

\end{description}

\sphinxlineitem{Returns}\begin{description}
\sphinxlineitem{None.}
\end{description}

\end{description}\end{quote}

\end{fulllineitems}



\subsection{Attributes}
\label{\detokenize{model:attributes}}\index{K (pywfe.model.Model attribute)@\spxentry{K}\spxextra{pywfe.model.Model attribute}}

\begin{fulllineitems}
\phantomsection\label{\detokenize{model:pywfe.model.Model.K}}
\pysigstartsignatures
\pysigline{\sphinxcode{\sphinxupquote{Model.}}\sphinxbfcode{\sphinxupquote{K}}}
\pysigstopsignatures
\sphinxAtStartPar
Sorted stiffness matrix

\end{fulllineitems}

\index{M (pywfe.model.Model attribute)@\spxentry{M}\spxextra{pywfe.model.Model attribute}}

\begin{fulllineitems}
\phantomsection\label{\detokenize{model:pywfe.model.Model.M}}
\pysigstartsignatures
\pysigline{\sphinxcode{\sphinxupquote{Model.}}\sphinxbfcode{\sphinxupquote{M}}}
\pysigstopsignatures
\sphinxAtStartPar
Sorted mass matrix

\end{fulllineitems}

\index{dof (pywfe.model.Model attribute)@\spxentry{dof}\spxextra{pywfe.model.Model attribute}}

\begin{fulllineitems}
\phantomsection\label{\detokenize{model:pywfe.model.Model.dof}}
\pysigstartsignatures
\pysigline{\sphinxcode{\sphinxupquote{Model.}}\sphinxbfcode{\sphinxupquote{dof}}}
\pysigstopsignatures
\sphinxAtStartPar
Sorted dof dictionary

\end{fulllineitems}

\index{node (pywfe.model.Model attribute)@\spxentry{node}\spxextra{pywfe.model.Model attribute}}

\begin{fulllineitems}
\phantomsection\label{\detokenize{model:pywfe.model.Model.node}}
\pysigstartsignatures
\pysigline{\sphinxcode{\sphinxupquote{Model.}}\sphinxbfcode{\sphinxupquote{node}}}
\pysigstopsignatures
\sphinxAtStartPar
dictionary of node information

\end{fulllineitems}

\index{K\_sub (pywfe.model.Model attribute)@\spxentry{K\_sub}\spxextra{pywfe.model.Model attribute}}

\begin{fulllineitems}
\phantomsection\label{\detokenize{model:pywfe.model.Model.K_sub}}
\pysigstartsignatures
\pysigline{\sphinxcode{\sphinxupquote{Model.}}\sphinxbfcode{\sphinxupquote{K\_sub}}}
\pysigstopsignatures
\sphinxAtStartPar
dictionary containing substructured stiffness matrices \sphinxcode{\sphinxupquote{\textquotesingle{}LL\textquotesingle{}, \textquotesingle{}LR, \textquotesingle{}RL\textquotesingle{}, \textquotesingle{}RR\textquotesingle{}, \textquotesingle{}LI\textquotesingle{}, \textquotesingle{}IL\textquotesingle{}, \textquotesingle{}RI\textquotesingle{}, \textquotesingle{}IR\textquotesingle{}, \textquotesingle{}II\textquotesingle{}}}

\end{fulllineitems}

\index{M\_sub (pywfe.model.Model attribute)@\spxentry{M\_sub}\spxextra{pywfe.model.Model attribute}}

\begin{fulllineitems}
\phantomsection\label{\detokenize{model:pywfe.model.Model.M_sub}}
\pysigstartsignatures
\pysigline{\sphinxcode{\sphinxupquote{Model.}}\sphinxbfcode{\sphinxupquote{M\_sub}}}
\pysigstopsignatures
\sphinxAtStartPar
dictionary containing substructured mass matrices.

\end{fulllineitems}

\index{eigensolution (pywfe.model.Model attribute)@\spxentry{eigensolution}\spxextra{pywfe.model.Model attribute}}

\begin{fulllineitems}
\phantomsection\label{\detokenize{model:pywfe.model.Model.eigensolution}}
\pysigstartsignatures
\pysigline{\sphinxcode{\sphinxupquote{Model.}}\sphinxbfcode{\sphinxupquote{eigensolution}}}
\pysigstopsignatures
\sphinxAtStartPar
The eigensolution at a given frequency. Gives values and vectors corresponding to propagation constants and mode shapes

\end{fulllineitems}

\index{force (pywfe.model.Model attribute)@\spxentry{force}\spxextra{pywfe.model.Model attribute}}

\begin{fulllineitems}
\phantomsection\label{\detokenize{model:pywfe.model.Model.force}}
\pysigstartsignatures
\pysigline{\sphinxcode{\sphinxupquote{Model.}}\sphinxbfcode{\sphinxupquote{force}}}
\pysigstopsignatures
\sphinxAtStartPar
The force vector corresponding to forces at each dof

\end{fulllineitems}



\subsection{Methods}
\label{\detokenize{model:methods}}\index{Model (class in pywfe)@\spxentry{Model}\spxextra{class in pywfe}}

\begin{fulllineitems}
\phantomsection\label{\detokenize{model:pywfe.Model}}
\pysigstartsignatures
\pysiglinewithargsret{\sphinxbfcode{\sphinxupquote{class\DUrole{w}{  }}}\sphinxcode{\sphinxupquote{pywfe.}}\sphinxbfcode{\sphinxupquote{Model}}}{\emph{\DUrole{n}{K}}, \emph{\DUrole{n}{M}}, \emph{\DUrole{n}{dof}}, \emph{\DUrole{n}{null}\DUrole{o}{=}\DUrole{default_value}{None}}, \emph{\DUrole{n}{nullf}\DUrole{o}{=}\DUrole{default_value}{None}}, \emph{\DUrole{n}{axis}\DUrole{o}{=}\DUrole{default_value}{0}}, \emph{\DUrole{n}{logging\_level}\DUrole{o}{=}\DUrole{default_value}{20}}, \emph{\DUrole{n}{solver}\DUrole{o}{=}\DUrole{default_value}{\textquotesingle{}transfer\_matrix\textquotesingle{}}}}{}
\pysigstopsignatures
\sphinxAtStartPar
The main high level api in the pywfe package.
\subsubsection*{Methods}


\begin{savenotes}\sphinxattablestart
\centering
\begin{tabulary}{\linewidth}[t]{\X{1}{2}\X{1}{2}}
\hline

\sphinxAtStartPar
{\hyperref[\detokenize{model:pywfe.Model.dispersion_relation}]{\sphinxcrossref{\sphinxcode{\sphinxupquote{dispersion\_relation}}}}}(frequency\_array{[}, ...{]})
&
\sphinxAtStartPar
Calculate frequency\sphinxhyphen{}wavenumber relation
\\
\hline
\sphinxAtStartPar
{\hyperref[\detokenize{model:pywfe.Model.displacements}]{\sphinxcrossref{\sphinxcode{\sphinxupquote{displacements}}}}}(x\_r{[}, f, dofs{]})
&
\sphinxAtStartPar
gets the displacements for all degrees of freedom at specified x and f.
\\
\hline
\sphinxAtStartPar
{\hyperref[\detokenize{model:pywfe.Model.dofs_to_indices}]{\sphinxcrossref{\sphinxcode{\sphinxupquote{dofs\_to\_indices}}}}}(dofs)
&
\sphinxAtStartPar
Generates indices for selected dofs
\\
\hline
\sphinxAtStartPar
{\hyperref[\detokenize{model:pywfe.Model.excited_amplitudes}]{\sphinxcrossref{\sphinxcode{\sphinxupquote{excited\_amplitudes}}}}}({[}f{]})
&
\sphinxAtStartPar
Find the excited amplitudes subject to a given force and frequency.
\\
\hline
\sphinxAtStartPar
{\hyperref[\detokenize{model:pywfe.Model.forces}]{\sphinxcrossref{\sphinxcode{\sphinxupquote{forces}}}}}(x\_r{[}, f, dofs{]})
&
\sphinxAtStartPar
Gets the total force on each degree of freedom.
\\
\hline
\sphinxAtStartPar
{\hyperref[\detokenize{model:pywfe.Model.form_dsm}]{\sphinxcrossref{\sphinxcode{\sphinxupquote{form\_dsm}}}}}(f)
&
\sphinxAtStartPar
Forms the DSM of the model at a given frequency
\\
\hline
\sphinxAtStartPar
{\hyperref[\detokenize{model:pywfe.Model.frequency_sweep}]{\sphinxcrossref{\sphinxcode{\sphinxupquote{frequency\_sweep}}}}}(f\_arr{[}, x\_r, quantities, ...{]})
&
\sphinxAtStartPar
Solves various quantities over specified frequency and response range.
\\
\hline
\sphinxAtStartPar
{\hyperref[\detokenize{model:pywfe.Model.generate_eigensolution}]{\sphinxcrossref{\sphinxcode{\sphinxupquote{generate\_eigensolution}}}}}(f)
&
\sphinxAtStartPar
Generates the sorted eigensolution at a given frequency.
\\
\hline
\sphinxAtStartPar
{\hyperref[\detokenize{model:pywfe.Model.left_dofs}]{\sphinxcrossref{\sphinxcode{\sphinxupquote{left\_dofs}}}}}()
&
\sphinxAtStartPar
get the dofs on the left face of the segment
\\
\hline
\sphinxAtStartPar
{\hyperref[\detokenize{model:pywfe.Model.modal_displacements}]{\sphinxcrossref{\sphinxcode{\sphinxupquote{modal\_displacements}}}}}(x\_r{[}, f, dofs{]})
&
\sphinxAtStartPar
Calculate the modal displacements at a given distance and frequency.
\\
\hline
\sphinxAtStartPar
{\hyperref[\detokenize{model:pywfe.Model.modal_forces}]{\sphinxcrossref{\sphinxcode{\sphinxupquote{modal\_forces}}}}}(x\_r{[}, f, dofs{]})
&
\sphinxAtStartPar
Generates the modal forces at given distance and frequency
\\
\hline
\sphinxAtStartPar
{\hyperref[\detokenize{model:pywfe.Model.phase_velocity}]{\sphinxcrossref{\sphinxcode{\sphinxupquote{phase\_velocity}}}}}(frequency\_array{[}, direction, ...{]})
&
\sphinxAtStartPar
gets the phase velocity curves for a given frequency array
\\
\hline
\sphinxAtStartPar
{\hyperref[\detokenize{model:pywfe.Model.propagated_amplitudes}]{\sphinxcrossref{\sphinxcode{\sphinxupquote{propagated\_amplitudes}}}}}(x\_r{[}, f{]})
&
\sphinxAtStartPar
Calculate the propagated and superimposed amplitudes for a given distance and frequency.
\\
\hline
\sphinxAtStartPar
{\hyperref[\detokenize{model:pywfe.Model.save}]{\sphinxcrossref{\sphinxcode{\sphinxupquote{save}}}}}(folder{[}, source{]})
&
\sphinxAtStartPar
Save the model to a folder
\\
\hline
\sphinxAtStartPar
{\hyperref[\detokenize{model:pywfe.Model.see}]{\sphinxcrossref{\sphinxcode{\sphinxupquote{see}}}}}()
&
\sphinxAtStartPar
Creates interactive matplotlib widget to visualise mesh and  inspect degrees of freedom.
\\
\hline
\sphinxAtStartPar
{\hyperref[\detokenize{model:pywfe.Model.select_dofs}]{\sphinxcrossref{\sphinxcode{\sphinxupquote{select\_dofs}}}}}({[}fieldvar{]})
&
\sphinxAtStartPar
select the model degrees of freedom that correspond to specified  field variable.
\\
\hline
\sphinxAtStartPar
{\hyperref[\detokenize{model:pywfe.Model.selection_index}]{\sphinxcrossref{\sphinxcode{\sphinxupquote{selection\_index}}}}}(dof)
&
\sphinxAtStartPar
Get the dof indices for a given selection.
\\
\hline
\sphinxAtStartPar
{\hyperref[\detokenize{model:pywfe.Model.transfer_function}]{\sphinxcrossref{\sphinxcode{\sphinxupquote{transfer\_function}}}}}(f\_arr, x\_r{[}, dofs, derivative{]})
&
\sphinxAtStartPar
Gets the displacement over frequency at specified distance and dofs.
\\
\hline
\sphinxAtStartPar
{\hyperref[\detokenize{model:pywfe.Model.wavenumbers}]{\sphinxcrossref{\sphinxcode{\sphinxupquote{wavenumbers}}}}}({[}f, direction, imag\_threshold{]})
&
\sphinxAtStartPar
Calculates the wavenumbers of the system at a given frequency
\\
\hline
\end{tabulary}
\par
\sphinxattableend\end{savenotes}
\index{solver (pywfe.Model attribute)@\spxentry{solver}\spxextra{pywfe.Model attribute}}

\begin{fulllineitems}
\phantomsection\label{\detokenize{model:pywfe.Model.solver}}
\pysigstartsignatures
\pysigline{\sphinxbfcode{\sphinxupquote{solver}}}
\pysigstopsignatures
\sphinxAtStartPar
Description of solver.

\end{fulllineitems}

\index{K (pywfe.Model attribute)@\spxentry{K}\spxextra{pywfe.Model attribute}}

\begin{fulllineitems}
\phantomsection\label{\detokenize{model:pywfe.Model.K}}
\pysigstartsignatures
\pysigline{\sphinxbfcode{\sphinxupquote{K}}}
\pysigstopsignatures
\sphinxAtStartPar
Sorted stiffness matrix

\end{fulllineitems}

\index{M (pywfe.Model attribute)@\spxentry{M}\spxextra{pywfe.Model attribute}}

\begin{fulllineitems}
\phantomsection\label{\detokenize{model:pywfe.Model.M}}
\pysigstartsignatures
\pysigline{\sphinxbfcode{\sphinxupquote{M}}}
\pysigstopsignatures
\sphinxAtStartPar
Sorted mass matrix

\end{fulllineitems}

\index{dof (pywfe.Model attribute)@\spxentry{dof}\spxextra{pywfe.Model attribute}}

\begin{fulllineitems}
\phantomsection\label{\detokenize{model:pywfe.Model.dof}}
\pysigstartsignatures
\pysigline{\sphinxbfcode{\sphinxupquote{dof}}}
\pysigstopsignatures
\sphinxAtStartPar
Sorted dof dictionary

\end{fulllineitems}

\index{K\_sub (pywfe.Model attribute)@\spxentry{K\_sub}\spxextra{pywfe.Model attribute}}

\begin{fulllineitems}
\phantomsection\label{\detokenize{model:pywfe.Model.K_sub}}
\pysigstartsignatures
\pysigline{\sphinxbfcode{\sphinxupquote{K\_sub}}}
\pysigstopsignatures
\sphinxAtStartPar
dictionary containing substructured stiffness matrices \sphinxcode{\sphinxupquote{\textquotesingle{}LL\textquotesingle{}, \textquotesingle{}LR, \textquotesingle{}RL\textquotesingle{}, \textquotesingle{}RR\textquotesingle{}, \textquotesingle{}LI\textquotesingle{}, \textquotesingle{}IL\textquotesingle{}, \textquotesingle{}RI\textquotesingle{}, \textquotesingle{}IR\textquotesingle{}, \textquotesingle{}II\textquotesingle{}}}

\end{fulllineitems}

\index{M\_sub (pywfe.Model attribute)@\spxentry{M\_sub}\spxextra{pywfe.Model attribute}}

\begin{fulllineitems}
\phantomsection\label{\detokenize{model:pywfe.Model.M_sub}}
\pysigstartsignatures
\pysigline{\sphinxbfcode{\sphinxupquote{M\_sub}}}
\pysigstopsignatures
\sphinxAtStartPar
dictionary containing substructured mass matrices.

\end{fulllineitems}

\index{node (pywfe.Model attribute)@\spxentry{node}\spxextra{pywfe.Model attribute}}

\begin{fulllineitems}
\phantomsection\label{\detokenize{model:pywfe.Model.node}}
\pysigstartsignatures
\pysigline{\sphinxbfcode{\sphinxupquote{node}}}
\pysigstopsignatures
\sphinxAtStartPar
dictionary of node information

\end{fulllineitems}

\index{delta (pywfe.Model attribute)@\spxentry{delta}\spxextra{pywfe.Model attribute}}

\begin{fulllineitems}
\phantomsection\label{\detokenize{model:pywfe.Model.delta}}
\pysigstartsignatures
\pysigline{\sphinxbfcode{\sphinxupquote{delta}}}
\pysigstopsignatures
\sphinxAtStartPar
Waveguide segment length

\end{fulllineitems}

\index{N (pywfe.Model attribute)@\spxentry{N}\spxextra{pywfe.Model attribute}}

\begin{fulllineitems}
\phantomsection\label{\detokenize{model:pywfe.Model.N}}
\pysigstartsignatures
\pysigline{\sphinxbfcode{\sphinxupquote{N}}}
\pysigstopsignatures
\sphinxAtStartPar
Number of dofs on both left and right faces combined

\end{fulllineitems}

\index{eigensolution (pywfe.Model attribute)@\spxentry{eigensolution}\spxextra{pywfe.Model attribute}}

\begin{fulllineitems}
\phantomsection\label{\detokenize{model:pywfe.Model.eigensolution}}
\pysigstartsignatures
\pysigline{\sphinxbfcode{\sphinxupquote{eigensolution}}}
\pysigstopsignatures
\sphinxAtStartPar
The eigensolution at a given frequency. Gives values and vectors corresponding to propagation constants and mode shapes

\end{fulllineitems}

\index{force (pywfe.Model attribute)@\spxentry{force}\spxextra{pywfe.Model attribute}}

\begin{fulllineitems}
\phantomsection\label{\detokenize{model:pywfe.Model.force}}
\pysigstartsignatures
\pysigline{\sphinxbfcode{\sphinxupquote{force}}}
\pysigstopsignatures
\sphinxAtStartPar
The force vector corresponding to forces at each dof

\end{fulllineitems}

\index{dofs\_to\_indices() (pywfe.Model method)@\spxentry{dofs\_to\_indices()}\spxextra{pywfe.Model method}}

\begin{fulllineitems}
\phantomsection\label{\detokenize{model:pywfe.Model.dofs_to_indices}}
\pysigstartsignatures
\pysiglinewithargsret{\sphinxbfcode{\sphinxupquote{dofs\_to\_indices}}}{\emph{\DUrole{n}{dofs}}}{}
\pysigstopsignatures
\sphinxAtStartPar
Generates indices for selected dofs
\begin{quote}\begin{description}
\sphinxlineitem{Parameters}\begin{description}
\sphinxlineitem{\sphinxstylestrong{dofs}}{[}str or list or dict{]}
\sphinxAtStartPar
‘all’ specifies all dofs.
A list of integers is interpreted as the dof indices.
A dof dictionary, created with \sphinxtitleref{model.select\_dofs()}

\end{description}

\sphinxlineitem{Returns}\begin{description}
\sphinxlineitem{\sphinxstylestrong{inds}}{[}np.ndarray{]}
\sphinxAtStartPar
array of dof indices.

\end{description}

\end{description}\end{quote}

\end{fulllineitems}

\index{form\_dsm() (pywfe.Model method)@\spxentry{form\_dsm()}\spxextra{pywfe.Model method}}

\begin{fulllineitems}
\phantomsection\label{\detokenize{model:pywfe.Model.form_dsm}}
\pysigstartsignatures
\pysiglinewithargsret{\sphinxbfcode{\sphinxupquote{form\_dsm}}}{\emph{\DUrole{n}{f}}}{}
\pysigstopsignatures
\sphinxAtStartPar
Forms the DSM of the model at a given frequency
\begin{quote}\begin{description}
\sphinxlineitem{Parameters}\begin{description}
\sphinxlineitem{\sphinxstylestrong{f}}{[}float{]}
\sphinxAtStartPar
frequency at which to form the DSM.

\end{description}

\sphinxlineitem{Returns}\begin{description}
\sphinxlineitem{\sphinxstylestrong{DSM}}{[}ndarray{]}
\sphinxAtStartPar
(ndof, ndof) sized array of type complex. The condensed DSM.

\end{description}

\end{description}\end{quote}

\end{fulllineitems}

\index{generate\_eigensolution() (pywfe.Model method)@\spxentry{generate\_eigensolution()}\spxextra{pywfe.Model method}}

\begin{fulllineitems}
\phantomsection\label{\detokenize{model:pywfe.Model.generate_eigensolution}}
\pysigstartsignatures
\pysiglinewithargsret{\sphinxbfcode{\sphinxupquote{generate\_eigensolution}}}{\emph{\DUrole{n}{f}}}{}
\pysigstopsignatures
\sphinxAtStartPar
Generates the sorted eigensolution at a given frequency.
If frequency is None or the presently calculated frequency,
then reuse the previously calculated eigensolution.
\begin{quote}\begin{description}
\sphinxlineitem{Parameters}\begin{description}
\sphinxlineitem{\sphinxstylestrong{f}}{[}float{]}
\sphinxAtStartPar
The frequency at which to calculate the eigensolution.

\end{description}

\sphinxlineitem{Returns}\begin{description}
\sphinxlineitem{\sphinxstylestrong{eigensolution}}{[}Eigensolution (namedtuple){]}\begin{description}
\sphinxlineitem{The sorted eigensolution. The named tuple fields are:}\begin{itemize}
\item {} 
\sphinxAtStartPar
lambda\_{[}plus{]}/{[}minus{]} : +/\sphinxhyphen{} going eigenvalues

\item {} 
\sphinxAtStartPar
phi\_{[}plus{]}/{[}minus{]} : +/\sphinxhyphen{} going right eigenvectors

\item {} 
\sphinxAtStartPar
psi\_{[}plus{]}/{[}minus{]} : +/\sphinxhyphen{} going left eigenvectors

\end{itemize}

\end{description}

\end{description}

\end{description}\end{quote}

\end{fulllineitems}

\index{wavenumbers() (pywfe.Model method)@\spxentry{wavenumbers()}\spxextra{pywfe.Model method}}

\begin{fulllineitems}
\phantomsection\label{\detokenize{model:pywfe.Model.wavenumbers}}
\pysigstartsignatures
\pysiglinewithargsret{\sphinxbfcode{\sphinxupquote{wavenumbers}}}{\emph{\DUrole{n}{f}\DUrole{o}{=}\DUrole{default_value}{None}}, \emph{\DUrole{n}{direction}\DUrole{o}{=}\DUrole{default_value}{\textquotesingle{}plus\textquotesingle{}}}, \emph{\DUrole{n}{imag\_threshold}\DUrole{o}{=}\DUrole{default_value}{None}}}{}
\pysigstopsignatures
\sphinxAtStartPar
Calculates the wavenumbers of the system at a given frequency
\begin{quote}\begin{description}
\sphinxlineitem{Parameters}\begin{description}
\sphinxlineitem{\sphinxstylestrong{f}}{[}float, optional{]}
\sphinxAtStartPar
Frequency at which to calculated wavenumbers. The default is None.

\sphinxlineitem{\sphinxstylestrong{direction}}{[}str, optional{]}
\sphinxAtStartPar
Choose positive going or negative going waves. The default is “plus”.

\sphinxlineitem{\sphinxstylestrong{imag\_threshold}}{[}float, optional{]}
\sphinxAtStartPar
Imaginary part of wavenumber above which will be set to np.nan.
The default is None.

\end{description}

\sphinxlineitem{Returns}\begin{description}
\sphinxlineitem{\sphinxstylestrong{k}}{[}ndarray{]}
\sphinxAtStartPar
The array of wavenumbers at this frequency.

\end{description}

\end{description}\end{quote}

\end{fulllineitems}

\index{dispersion\_relation() (pywfe.Model method)@\spxentry{dispersion\_relation()}\spxextra{pywfe.Model method}}

\begin{fulllineitems}
\phantomsection\label{\detokenize{model:pywfe.Model.dispersion_relation}}
\pysigstartsignatures
\pysiglinewithargsret{\sphinxbfcode{\sphinxupquote{dispersion\_relation}}}{\emph{\DUrole{n}{frequency\_array}}, \emph{\DUrole{n}{direction}\DUrole{o}{=}\DUrole{default_value}{\textquotesingle{}plus\textquotesingle{}}}, \emph{\DUrole{n}{imag\_threshold}\DUrole{o}{=}\DUrole{default_value}{None}}}{}
\pysigstopsignatures
\sphinxAtStartPar
Calculate frequency\sphinxhyphen{}wavenumber relation
\begin{quote}\begin{description}
\sphinxlineitem{Parameters}\begin{description}
\sphinxlineitem{\sphinxstylestrong{frequency\_array}}{[}ndarray{]}
\sphinxAtStartPar
Frequencies to calculate.

\sphinxlineitem{\sphinxstylestrong{direction}}{[}str, optional{]}
\sphinxAtStartPar
Choose positive going or negative going waves. The default is “plus”.

\sphinxlineitem{\sphinxstylestrong{imag\_threshold}}{[}float, optional{]}
\sphinxAtStartPar
Imaginary part of wavenumber above which will be set to np.nan.
The default is None.

\end{description}

\sphinxlineitem{Returns}\begin{description}
\sphinxlineitem{\sphinxstylestrong{wavenumbers}}{[}ndarray{]}
\sphinxAtStartPar
(nfreq, n\_waves) sized array of type complex.

\end{description}

\end{description}\end{quote}

\end{fulllineitems}

\index{phase\_velocity() (pywfe.Model method)@\spxentry{phase\_velocity()}\spxextra{pywfe.Model method}}

\begin{fulllineitems}
\phantomsection\label{\detokenize{model:pywfe.Model.phase_velocity}}
\pysigstartsignatures
\pysiglinewithargsret{\sphinxbfcode{\sphinxupquote{phase\_velocity}}}{\emph{\DUrole{n}{frequency\_array}}, \emph{\DUrole{n}{direction}\DUrole{o}{=}\DUrole{default_value}{\textquotesingle{}plus\textquotesingle{}}}, \emph{\DUrole{n}{imag\_threshold}\DUrole{o}{=}\DUrole{default_value}{None}}}{}
\pysigstopsignatures
\sphinxAtStartPar
gets the phase velocity curves for a given frequency array
\begin{quote}\begin{description}
\sphinxlineitem{Parameters}\begin{description}
\sphinxlineitem{\sphinxstylestrong{frequency\_array}}{[}np.ndarray{]}
\sphinxAtStartPar
DESCRIPTION.

\sphinxlineitem{\sphinxstylestrong{direction}}{[}str, optional{]}
\sphinxAtStartPar
Direction of the waves. The default is ‘plus’.

\sphinxlineitem{\sphinxstylestrong{imag\_threshold}}{[}float, optional{]}
\sphinxAtStartPar
Imaginary threshold above which set to np.nan. The default is None.

\end{description}

\sphinxlineitem{Returns}\begin{description}
\sphinxlineitem{ndarray}
\sphinxAtStartPar
phase velocity.

\end{description}

\end{description}\end{quote}

\end{fulllineitems}

\index{excited\_amplitudes() (pywfe.Model method)@\spxentry{excited\_amplitudes()}\spxextra{pywfe.Model method}}

\begin{fulllineitems}
\phantomsection\label{\detokenize{model:pywfe.Model.excited_amplitudes}}
\pysigstartsignatures
\pysiglinewithargsret{\sphinxbfcode{\sphinxupquote{excited\_amplitudes}}}{\emph{\DUrole{n}{f}\DUrole{o}{=}\DUrole{default_value}{None}}}{}
\pysigstopsignatures
\sphinxAtStartPar
Find the excited amplitudes subject to a given force and frequency.
If the solution has already been calculated for the same inputs,
reuse the old solution.
\begin{quote}\begin{description}
\sphinxlineitem{Parameters}\begin{description}
\sphinxlineitem{\sphinxstylestrong{f}}{[}float, optional{]}
\sphinxAtStartPar
Frequency. The default is None.

\end{description}

\sphinxlineitem{Returns}\begin{description}
\sphinxlineitem{\sphinxstylestrong{e\_plus}}{[}ndarray{]}
\sphinxAtStartPar
Positive excited wave amplitudes.

\sphinxlineitem{\sphinxstylestrong{e\_minus}}{[}ndarray{]}
\sphinxAtStartPar
Negative excited wave amplitudes.

\end{description}

\end{description}\end{quote}

\end{fulllineitems}

\index{propagated\_amplitudes() (pywfe.Model method)@\spxentry{propagated\_amplitudes()}\spxextra{pywfe.Model method}}

\begin{fulllineitems}
\phantomsection\label{\detokenize{model:pywfe.Model.propagated_amplitudes}}
\pysigstartsignatures
\pysiglinewithargsret{\sphinxbfcode{\sphinxupquote{propagated\_amplitudes}}}{\emph{\DUrole{n}{x\_r}}, \emph{\DUrole{n}{f}\DUrole{o}{=}\DUrole{default_value}{None}}}{}
\pysigstopsignatures
\sphinxAtStartPar
Calculate the propagated and superimposed amplitudes
for a given distance and frequency.
\begin{quote}\begin{description}
\sphinxlineitem{Parameters}\begin{description}
\sphinxlineitem{\sphinxstylestrong{x\_r}}{[}float{]}
\sphinxAtStartPar
Axial response distance.

\sphinxlineitem{\sphinxstylestrong{f}}{[}float, optional{]}
\sphinxAtStartPar
Frequency. The default is None.

\end{description}

\sphinxlineitem{Returns}\begin{description}
\sphinxlineitem{\sphinxstylestrong{b\_plus, b\_minus}}{[}ndarray{]}
\sphinxAtStartPar
Positive and negative amplitudes.

\end{description}

\end{description}\end{quote}

\end{fulllineitems}

\index{modal\_displacements() (pywfe.Model method)@\spxentry{modal\_displacements()}\spxextra{pywfe.Model method}}

\begin{fulllineitems}
\phantomsection\label{\detokenize{model:pywfe.Model.modal_displacements}}
\pysigstartsignatures
\pysiglinewithargsret{\sphinxbfcode{\sphinxupquote{modal\_displacements}}}{\emph{\DUrole{n}{x\_r}}, \emph{\DUrole{n}{f}\DUrole{o}{=}\DUrole{default_value}{None}}, \emph{\DUrole{n}{dofs}\DUrole{o}{=}\DUrole{default_value}{\textquotesingle{}all\textquotesingle{}}}}{}
\pysigstopsignatures
\sphinxAtStartPar
Calculate the modal displacements at a given distance and frequency.
Each column corresponds to a different wavemode, each row is
a different degree of freedom.
\begin{quote}\begin{description}
\sphinxlineitem{Parameters}\begin{description}
\sphinxlineitem{\sphinxstylestrong{x\_r}}{[}float{]}
\sphinxAtStartPar
Axial response distance.

\sphinxlineitem{\sphinxstylestrong{f}}{[}float, optional{]}
\sphinxAtStartPar
Frequency. The default is None.

\end{description}

\sphinxlineitem{Returns}\begin{description}
\sphinxlineitem{\sphinxstylestrong{q\_j\_plus, q\_j\_minus}}{[}ndarray{]}
\sphinxAtStartPar
The modal displacements for positive and negative going waves.

\end{description}

\end{description}\end{quote}

\end{fulllineitems}

\index{displacements() (pywfe.Model method)@\spxentry{displacements()}\spxextra{pywfe.Model method}}

\begin{fulllineitems}
\phantomsection\label{\detokenize{model:pywfe.Model.displacements}}
\pysigstartsignatures
\pysiglinewithargsret{\sphinxbfcode{\sphinxupquote{displacements}}}{\emph{\DUrole{n}{x\_r}}, \emph{\DUrole{n}{f}\DUrole{o}{=}\DUrole{default_value}{None}}, \emph{\DUrole{n}{dofs}\DUrole{o}{=}\DUrole{default_value}{\textquotesingle{}all\textquotesingle{}}}}{}
\pysigstopsignatures
\sphinxAtStartPar
gets the displacements for all degrees of freedom at specified x and f.
\begin{quote}\begin{description}
\sphinxlineitem{Parameters}\begin{description}
\sphinxlineitem{\sphinxstylestrong{x\_r}}{[}float{]}
\sphinxAtStartPar
response distance (can be array like).

\sphinxlineitem{\sphinxstylestrong{f}}{[}float, optional{]}
\sphinxAtStartPar
Frequency. The default is None.

\end{description}

\sphinxlineitem{Returns}\begin{description}
\sphinxlineitem{ndarray}
\sphinxAtStartPar
displacements for each degree of freedom.

\end{description}

\end{description}\end{quote}

\end{fulllineitems}

\index{modal\_forces() (pywfe.Model method)@\spxentry{modal\_forces()}\spxextra{pywfe.Model method}}

\begin{fulllineitems}
\phantomsection\label{\detokenize{model:pywfe.Model.modal_forces}}
\pysigstartsignatures
\pysiglinewithargsret{\sphinxbfcode{\sphinxupquote{modal\_forces}}}{\emph{\DUrole{n}{x\_r}}, \emph{\DUrole{n}{f}\DUrole{o}{=}\DUrole{default_value}{None}}, \emph{\DUrole{n}{dofs}\DUrole{o}{=}\DUrole{default_value}{\textquotesingle{}all\textquotesingle{}}}}{}
\pysigstopsignatures
\sphinxAtStartPar
Generates the modal forces at given distance and frequency
\begin{quote}\begin{description}
\sphinxlineitem{Parameters}\begin{description}
\sphinxlineitem{\sphinxstylestrong{x\_r}}{[}float{]}
\sphinxAtStartPar
Response distance.

\sphinxlineitem{\sphinxstylestrong{f}}{[}float, optional{]}
\sphinxAtStartPar
Frequency. The default is None.

\end{description}

\sphinxlineitem{Returns}\begin{description}
\sphinxlineitem{np.ndarray}
\sphinxAtStartPar
modal force array.

\end{description}

\end{description}\end{quote}

\end{fulllineitems}

\index{forces() (pywfe.Model method)@\spxentry{forces()}\spxextra{pywfe.Model method}}

\begin{fulllineitems}
\phantomsection\label{\detokenize{model:pywfe.Model.forces}}
\pysigstartsignatures
\pysiglinewithargsret{\sphinxbfcode{\sphinxupquote{forces}}}{\emph{\DUrole{n}{x\_r}}, \emph{\DUrole{n}{f}\DUrole{o}{=}\DUrole{default_value}{None}}, \emph{\DUrole{n}{dofs}\DUrole{o}{=}\DUrole{default_value}{\textquotesingle{}all\textquotesingle{}}}}{}
\pysigstopsignatures
\sphinxAtStartPar
Gets the total force on each degree of freedom.
\begin{quote}\begin{description}
\sphinxlineitem{Parameters}\begin{description}
\sphinxlineitem{\sphinxstylestrong{x\_r}}{[}float{]}
\sphinxAtStartPar
Response distance.

\sphinxlineitem{\sphinxstylestrong{f}}{[}float, optional{]}
\sphinxAtStartPar
Frequency. The default is None.

\end{description}

\sphinxlineitem{Returns}\begin{description}
\sphinxlineitem{np.ndarray}
\sphinxAtStartPar
forces.

\end{description}

\end{description}\end{quote}

\end{fulllineitems}

\index{frequency\_sweep() (pywfe.Model method)@\spxentry{frequency\_sweep()}\spxextra{pywfe.Model method}}

\begin{fulllineitems}
\phantomsection\label{\detokenize{model:pywfe.Model.frequency_sweep}}
\pysigstartsignatures
\pysiglinewithargsret{\sphinxbfcode{\sphinxupquote{frequency\_sweep}}}{\emph{\DUrole{n}{f\_arr}}, \emph{\DUrole{n}{x\_r}\DUrole{o}{=}\DUrole{default_value}{0}}, \emph{\DUrole{n}{quantities}\DUrole{o}{=}\DUrole{default_value}{{[}\textquotesingle{}displacements\textquotesingle{}{]}}}, \emph{\DUrole{n}{mac}\DUrole{o}{=}\DUrole{default_value}{False}}, \emph{\DUrole{n}{dofs}\DUrole{o}{=}\DUrole{default_value}{\textquotesingle{}all\textquotesingle{}}}}{}
\pysigstopsignatures
\sphinxAtStartPar
Solves various quantities over specified frequency and response range.
Includes Modal Assurance Critereon (MAC) sorting.
\begin{quote}\begin{description}
\sphinxlineitem{Parameters}\begin{description}
\sphinxlineitem{\sphinxstylestrong{f\_arr}}{[}np.ndarray{]}
\sphinxAtStartPar
Array of frequencies.

\sphinxlineitem{\sphinxstylestrong{x\_r}}{[}float or np.ndarray, optional{]}
\sphinxAtStartPar
Response distance. The default is 0.

\sphinxlineitem{\sphinxstylestrong{quantities}}{[}list, optional{]}
\sphinxAtStartPar
Quantities to solve for. The default is {[}‘displacements’{]}.

\sphinxlineitem{\sphinxstylestrong{mac}}{[}bool, optional{]}
\sphinxAtStartPar
Whether to sort modal quantities according to MAC. The default is False.

\sphinxlineitem{\sphinxstylestrong{dofs}}{[}list, optional{]}
\sphinxAtStartPar
Select specific degrees of freedom. The default is ‘all’.

\end{description}

\sphinxlineitem{Returns}\begin{description}
\sphinxlineitem{dict}
\sphinxAtStartPar
Dictionary of output for specified quantities.

\end{description}

\end{description}\end{quote}

\end{fulllineitems}

\index{transfer\_function() (pywfe.Model method)@\spxentry{transfer\_function()}\spxextra{pywfe.Model method}}

\begin{fulllineitems}
\phantomsection\label{\detokenize{model:pywfe.Model.transfer_function}}
\pysigstartsignatures
\pysiglinewithargsret{\sphinxbfcode{\sphinxupquote{transfer\_function}}}{\emph{\DUrole{n}{f\_arr}}, \emph{\DUrole{n}{x\_r}}, \emph{\DUrole{n}{dofs}\DUrole{o}{=}\DUrole{default_value}{\textquotesingle{}all\textquotesingle{}}}, \emph{\DUrole{n}{derivative}\DUrole{o}{=}\DUrole{default_value}{0}}}{}
\pysigstopsignatures
\sphinxAtStartPar
Gets the displacement over frequency at specified distance and dofs.
\begin{quote}\begin{description}
\sphinxlineitem{Parameters}\begin{description}
\sphinxlineitem{\sphinxstylestrong{f\_arr}}{[}np.ndarray{]}
\sphinxAtStartPar
Frequency array.

\sphinxlineitem{\sphinxstylestrong{x\_r}}{[}float or np.ndarray{]}
\sphinxAtStartPar
Response distance.

\sphinxlineitem{\sphinxstylestrong{dofs}}{[}list, optional{]}
\sphinxAtStartPar
List of dofs to return. The default is “all”.

\end{description}

\sphinxlineitem{Returns}\begin{description}
\sphinxlineitem{ndarray}
\sphinxAtStartPar
Displacements over frequency and distance.

\end{description}

\end{description}\end{quote}

\end{fulllineitems}

\index{select\_dofs() (pywfe.Model method)@\spxentry{select\_dofs()}\spxextra{pywfe.Model method}}

\begin{fulllineitems}
\phantomsection\label{\detokenize{model:pywfe.Model.select_dofs}}
\pysigstartsignatures
\pysiglinewithargsret{\sphinxbfcode{\sphinxupquote{select\_dofs}}}{\emph{\DUrole{n}{fieldvar}\DUrole{o}{=}\DUrole{default_value}{None}}}{}
\pysigstopsignatures
\sphinxAtStartPar
select the model degrees of freedom that correspond to specified 
field variable.
\begin{quote}\begin{description}
\sphinxlineitem{Parameters}\begin{description}
\sphinxlineitem{\sphinxstylestrong{fieldvar}}{[}str or list, optional{]}
\sphinxAtStartPar
The fieldvariable or list thereof to select for. The default is None.

\end{description}

\sphinxlineitem{Returns}\begin{description}
\sphinxlineitem{\sphinxstylestrong{dofs}}{[}dict{]}
\sphinxAtStartPar
Reduced dof dictionary.

\end{description}

\end{description}\end{quote}

\end{fulllineitems}

\index{left\_dofs() (pywfe.Model method)@\spxentry{left\_dofs()}\spxextra{pywfe.Model method}}

\begin{fulllineitems}
\phantomsection\label{\detokenize{model:pywfe.Model.left_dofs}}
\pysigstartsignatures
\pysiglinewithargsret{\sphinxbfcode{\sphinxupquote{left\_dofs}}}{}{}
\pysigstopsignatures
\sphinxAtStartPar
get the dofs on the left face of the segment
\begin{quote}\begin{description}
\sphinxlineitem{Returns}\begin{description}
\sphinxlineitem{\sphinxstylestrong{dofs}}{[}dict{]}
\sphinxAtStartPar
dof dictionary.

\end{description}

\end{description}\end{quote}

\end{fulllineitems}

\index{selection\_index() (pywfe.Model method)@\spxentry{selection\_index()}\spxextra{pywfe.Model method}}

\begin{fulllineitems}
\phantomsection\label{\detokenize{model:pywfe.Model.selection_index}}
\pysigstartsignatures
\pysiglinewithargsret{\sphinxbfcode{\sphinxupquote{selection\_index}}}{\emph{\DUrole{n}{dof}}}{}
\pysigstopsignatures
\sphinxAtStartPar
Get the dof indices for a given selection.
\begin{quote}\begin{description}
\sphinxlineitem{Parameters}\begin{description}
\sphinxlineitem{\sphinxstylestrong{dof}}{[}dict{]}
\sphinxAtStartPar
dof dictionary.

\end{description}

\sphinxlineitem{Returns}\begin{description}
\sphinxlineitem{np.ndarray}
\sphinxAtStartPar
1D array of indices for selected dofs.

\end{description}

\end{description}\end{quote}

\end{fulllineitems}

\index{see() (pywfe.Model method)@\spxentry{see()}\spxextra{pywfe.Model method}}

\begin{fulllineitems}
\phantomsection\label{\detokenize{model:pywfe.Model.see}}
\pysigstartsignatures
\pysiglinewithargsret{\sphinxbfcode{\sphinxupquote{see}}}{}{}
\pysigstopsignatures
\sphinxAtStartPar
Creates interactive matplotlib widget to visualise mesh and 
inspect degrees of freedom.
\begin{quote}\begin{description}
\sphinxlineitem{Returns}\begin{description}
\sphinxlineitem{None.}
\end{description}

\end{description}\end{quote}

\end{fulllineitems}

\index{save() (pywfe.Model method)@\spxentry{save()}\spxextra{pywfe.Model method}}

\begin{fulllineitems}
\phantomsection\label{\detokenize{model:pywfe.Model.save}}
\pysigstartsignatures
\pysiglinewithargsret{\sphinxbfcode{\sphinxupquote{save}}}{\emph{\DUrole{n}{folder}}, \emph{\DUrole{n}{source}\DUrole{o}{=}\DUrole{default_value}{\textquotesingle{}local\textquotesingle{}}}}{}
\pysigstopsignatures
\sphinxAtStartPar
Save the model to a folder
\begin{quote}\begin{description}
\sphinxlineitem{Parameters}\begin{description}
\sphinxlineitem{\sphinxstylestrong{folder}}{[}str{]}
\sphinxAtStartPar
folder name.

\sphinxlineitem{\sphinxstylestrong{source}}{[}str, optional{]}
\sphinxAtStartPar
Save to \sphinxcode{\sphinxupquote{\textquotesingle{}local\textquotesingle{}}} or \sphinxcode{\sphinxupquote{\textquotesingle{}database\textquotesingle{}}}. The default is ‘local’.

\end{description}

\sphinxlineitem{Returns}\begin{description}
\sphinxlineitem{None.}
\end{description}

\end{description}\end{quote}

\end{fulllineitems}


\end{fulllineitems}


\sphinxstepscope


\section{core}
\label{\detokenize{core:core}}\label{\detokenize{core::doc}}
\begin{sphinxShadowBox}
\begin{itemize}
\item {} 
\sphinxAtStartPar
\phantomsection\label{\detokenize{core:id1}}{\hyperref[\detokenize{core:model-setup}]{\sphinxcrossref{model\_setup}}}

\item {} 
\sphinxAtStartPar
\phantomsection\label{\detokenize{core:id2}}{\hyperref[\detokenize{core:eigensolvers}]{\sphinxcrossref{eigensolvers}}}

\item {} 
\sphinxAtStartPar
\phantomsection\label{\detokenize{core:id3}}{\hyperref[\detokenize{core:classify-modes}]{\sphinxcrossref{classify\_modes}}}

\item {} 
\sphinxAtStartPar
\phantomsection\label{\detokenize{core:id4}}{\hyperref[\detokenize{core:forced-problem}]{\sphinxcrossref{forced\_problem}}}

\end{itemize}
\end{sphinxShadowBox}
\phantomsection\label{\detokenize{core:module-pywfe.core.model_setup}}\index{module@\spxentry{module}!pywfe.core.model\_setup@\spxentry{pywfe.core.model\_setup}}\index{pywfe.core.model\_setup@\spxentry{pywfe.core.model\_setup}!module@\spxentry{module}}

\subsection{model\_setup}
\label{\detokenize{core:model-setup}}
\sphinxAtStartPar
This module contains the functions for setting up the WFE model.
\begin{description}
\sphinxlineitem{This includes:}\begin{itemize}
\item {} 
\sphinxAtStartPar
Creating the relevant \sphinxtitleref{dof} dict data

\item {} 
\sphinxAtStartPar
Applying the boundary conditions

\item {} 
\sphinxAtStartPar
Sorting M, K and dofs to left and right faces

\item {} 
\sphinxAtStartPar
Creating \sphinxtitleref{node} dict data

\end{itemize}

\end{description}
\index{generate\_dof\_info() (in module pywfe.core.model\_setup)@\spxentry{generate\_dof\_info()}\spxextra{in module pywfe.core.model\_setup}}

\begin{fulllineitems}
\phantomsection\label{\detokenize{core:pywfe.core.model_setup.generate_dof_info}}
\pysigstartsignatures
\pysiglinewithargsret{\sphinxcode{\sphinxupquote{pywfe.core.model\_setup.}}\sphinxbfcode{\sphinxupquote{generate\_dof\_info}}}{\emph{\DUrole{n}{dof}\DUrole{p}{:}\DUrole{w}{  }\DUrole{n}{dict}}, \emph{\DUrole{n}{axis}\DUrole{o}{=}\DUrole{default_value}{0}}}{}
\pysigstopsignatures
\sphinxAtStartPar
Generates the \sphinxtitleref{dof} dictionary, including which face each dof is on.
Also rolls sets the waveguide axis and created index array if none given.
\begin{quote}\begin{description}
\sphinxlineitem{Parameters}\begin{description}
\sphinxlineitem{\sphinxstylestrong{dof}}{[}dict{]}
\sphinxAtStartPar
DESCRIPTION.

\sphinxlineitem{\sphinxstylestrong{axis}}{[}TYPE, optional{]}
\sphinxAtStartPar
DESCRIPTION. The default is 0.

\end{description}

\sphinxlineitem{Returns}\begin{description}
\sphinxlineitem{\sphinxstylestrong{dof}}{[}TYPE{]}
\sphinxAtStartPar
DESCRIPTION.

\end{description}

\end{description}\end{quote}

\end{fulllineitems}

\index{apply\_boundary\_conditions() (in module pywfe.core.model\_setup)@\spxentry{apply\_boundary\_conditions()}\spxextra{in module pywfe.core.model\_setup}}

\begin{fulllineitems}
\phantomsection\label{\detokenize{core:pywfe.core.model_setup.apply_boundary_conditions}}
\pysigstartsignatures
\pysiglinewithargsret{\sphinxcode{\sphinxupquote{pywfe.core.model\_setup.}}\sphinxbfcode{\sphinxupquote{apply\_boundary\_conditions}}}{\emph{\DUrole{n}{K}}, \emph{\DUrole{n}{M}}, \emph{\DUrole{n}{dof}}, \emph{\DUrole{n}{null}}, \emph{\DUrole{n}{nullf}}}{}
\pysigstopsignatures
\sphinxAtStartPar
Applies boundary conditions according to null constraint matrices.
Resorts and removes degrees of freedom as needed. (NOT FINISHED)
\begin{quote}\begin{description}
\sphinxlineitem{Parameters}\begin{description}
\sphinxlineitem{\sphinxstylestrong{K}}{[}ndarray{]}
\sphinxAtStartPar
(ndof, ndof) sized array of type float or complex.

\sphinxlineitem{\sphinxstylestrong{M}}{[}ndarray{]}
\sphinxAtStartPar
(ndof, ndof) sized array of type float or complex.

\sphinxlineitem{\sphinxstylestrong{dof}}{[}dict{]}
\sphinxAtStartPar
dof dictionary.

\sphinxlineitem{\sphinxstylestrong{null}}{[}ndarray{]}
\sphinxAtStartPar
(ndof, ndof) sized array of type float.

\sphinxlineitem{\sphinxstylestrong{nullf}}{[}ndarray{]}
\sphinxAtStartPar
(ndof, ndof) sized array of type float.

\end{description}

\sphinxlineitem{Returns}\begin{description}
\sphinxlineitem{\sphinxstylestrong{K}}{[}ndarray{]}
\sphinxAtStartPar
(ndof, ndof) sized array of type float or complex.

\sphinxlineitem{\sphinxstylestrong{M}}{[}ndarray{]}
\sphinxAtStartPar
(ndof, ndof) sized array of type float or complex.

\sphinxlineitem{\sphinxstylestrong{dof}}{[}dict{]}
\sphinxAtStartPar
dof dictionary.

\end{description}

\end{description}\end{quote}

\end{fulllineitems}

\index{order\_system\_faces() (in module pywfe.core.model\_setup)@\spxentry{order\_system\_faces()}\spxextra{in module pywfe.core.model\_setup}}

\begin{fulllineitems}
\phantomsection\label{\detokenize{core:pywfe.core.model_setup.order_system_faces}}
\pysigstartsignatures
\pysiglinewithargsret{\sphinxcode{\sphinxupquote{pywfe.core.model\_setup.}}\sphinxbfcode{\sphinxupquote{order\_system\_faces}}}{\emph{\DUrole{n}{K}}, \emph{\DUrole{n}{M}}, \emph{\DUrole{n}{dof}}}{}
\pysigstopsignatures\begin{quote}\begin{description}
\sphinxlineitem{Parameters}\begin{description}
\sphinxlineitem{\sphinxstylestrong{K}}{[}ndarray{]}
\sphinxAtStartPar
(ndof, ndof) sized array of type float or complex.

\sphinxlineitem{\sphinxstylestrong{M}}{[}ndarray{]}
\sphinxAtStartPar
(ndof, ndof) sized array of type float or complex.

\sphinxlineitem{\sphinxstylestrong{dof}}{[}dict{]}
\sphinxAtStartPar
dof dictionary.

\end{description}

\sphinxlineitem{Returns}\begin{description}
\sphinxlineitem{\sphinxstylestrong{K}}{[}ndarray{]}
\sphinxAtStartPar
(ndof, ndof) sized array of type float or complex.

\sphinxlineitem{\sphinxstylestrong{M}}{[}ndarray{]}
\sphinxAtStartPar
(ndof, ndof) sized array of type float or complex.

\sphinxlineitem{\sphinxstylestrong{dof}}{[}dict{]}
\sphinxAtStartPar
dof dictionary.

\end{description}

\end{description}\end{quote}

\end{fulllineitems}

\index{substructure\_matrices() (in module pywfe.core.model\_setup)@\spxentry{substructure\_matrices()}\spxextra{in module pywfe.core.model\_setup}}

\begin{fulllineitems}
\phantomsection\label{\detokenize{core:pywfe.core.model_setup.substructure_matrices}}
\pysigstartsignatures
\pysiglinewithargsret{\sphinxcode{\sphinxupquote{pywfe.core.model\_setup.}}\sphinxbfcode{\sphinxupquote{substructure\_matrices}}}{\emph{\DUrole{n}{K}}, \emph{\DUrole{n}{M}}, \emph{\DUrole{n}{dof}}}{}
\pysigstopsignatures
\sphinxAtStartPar
Creates dictionaries for the submatrices of K and M
\begin{quote}\begin{description}
\sphinxlineitem{Parameters}\begin{description}
\sphinxlineitem{\sphinxstylestrong{K}}{[}ndarray{]}
\sphinxAtStartPar
(ndof, ndof) sized array of type float or complex.

\sphinxlineitem{\sphinxstylestrong{M}}{[}ndarray{]}
\sphinxAtStartPar
(ndof, ndof) sized array of type float or complex.

\sphinxlineitem{\sphinxstylestrong{dof}}{[}dict{]}
\sphinxAtStartPar
dof dictionary.

\end{description}

\sphinxlineitem{Returns}\begin{description}
\sphinxlineitem{\sphinxstylestrong{K\_sub}}{[}dict{]}
\sphinxAtStartPar
dictionary of substructured stiffness matrices.

\sphinxlineitem{\sphinxstylestrong{M\_sub}}{[}dict{]}
\sphinxAtStartPar
dictionary of substructured mass matrices.

\end{description}

\end{description}\end{quote}

\end{fulllineitems}

\index{create\_node\_dict() (in module pywfe.core.model\_setup)@\spxentry{create\_node\_dict()}\spxextra{in module pywfe.core.model\_setup}}

\begin{fulllineitems}
\phantomsection\label{\detokenize{core:pywfe.core.model_setup.create_node_dict}}
\pysigstartsignatures
\pysiglinewithargsret{\sphinxcode{\sphinxupquote{pywfe.core.model\_setup.}}\sphinxbfcode{\sphinxupquote{create\_node\_dict}}}{\emph{\DUrole{n}{dof}}}{}
\pysigstopsignatures
\sphinxAtStartPar
Creates node dictionary for nodes on the left face of the model
\begin{quote}\begin{description}
\sphinxlineitem{Parameters}\begin{description}
\sphinxlineitem{\sphinxstylestrong{dof}}{[}dict{]}
\sphinxAtStartPar
dof dictionary.

\end{description}

\sphinxlineitem{Returns}\begin{description}
\sphinxlineitem{\sphinxstylestrong{node}}{[}dict{]}
\sphinxAtStartPar
node dictionary.

\end{description}

\end{description}\end{quote}

\end{fulllineitems}

\phantomsection\label{\detokenize{core:module-pywfe.core.eigensolvers}}\index{module@\spxentry{module}!pywfe.core.eigensolvers@\spxentry{pywfe.core.eigensolvers}}\index{pywfe.core.eigensolvers@\spxentry{pywfe.core.eigensolvers}!module@\spxentry{module}}

\subsection{eigensolvers}
\label{\detokenize{core:eigensolvers}}
\sphinxAtStartPar
This module contains different solvers for the WFE eigenproblem.
\index{transfer\_matrix() (in module pywfe.core.eigensolvers)@\spxentry{transfer\_matrix()}\spxextra{in module pywfe.core.eigensolvers}}

\begin{fulllineitems}
\phantomsection\label{\detokenize{core:pywfe.core.eigensolvers.transfer_matrix}}
\pysigstartsignatures
\pysiglinewithargsret{\sphinxcode{\sphinxupquote{pywfe.core.eigensolvers.}}\sphinxbfcode{\sphinxupquote{transfer\_matrix}}}{\emph{\DUrole{n}{DSM}}}{}
\pysigstopsignatures
\sphinxAtStartPar
Classical transfer matrix formulation of the WFE eigenproblem.

\sphinxAtStartPar
The transfer function is defined as
\begin{equation*}
\begin{split}\mathbf{T} = \begin{bmatrix}
-D_{LR}^{-1} D_{LL} & D_{LR}^{-1} \\
-D_{RL}+D_{RR} D_{LR}^{-1} D_{LL} & -D_{RR} D_{LR}^{-1}
\end{bmatrix}\end{split}
\end{equation*}
\sphinxAtStartPar
which leads to the eigenvalue problem
\begin{equation*}
\begin{split}T \mathbf{\Phi} = \lambda \mathbf{\Phi}\end{split}
\end{equation*}
\sphinxAtStartPar
The left eigenvectors can be found by considering \(\mathbf{T}^{T}\)
\begin{quote}\begin{description}
\sphinxlineitem{Parameters}\begin{description}
\sphinxlineitem{\sphinxstylestrong{DSM}}{[}(N,N) ndarray (float or complex){]}
\sphinxAtStartPar
The dynamic stiffness matrix of the system. 
NxN array of type float or complex.

\end{description}

\sphinxlineitem{Returns}\begin{description}
\sphinxlineitem{\sphinxstylestrong{vals}}{[}ndarray{]}
\sphinxAtStartPar
1\sphinxhyphen{}D array of length N type complex.

\sphinxlineitem{\sphinxstylestrong{left\_eigenvectors}}{[}ndarray{]}
\sphinxAtStartPar
NxN array of type float or complex.
Column i is vector corresponding to vals{[}i{]}

\sphinxlineitem{\sphinxstylestrong{right\_eigenvectors}}{[}ndarray{]}
\sphinxAtStartPar
NxN array of type float or complex.
Column i is vector corresponding to vals{[}i{]}

\end{description}

\end{description}\end{quote}

\end{fulllineitems}

\index{polynomial() (in module pywfe.core.eigensolvers)@\spxentry{polynomial()}\spxextra{in module pywfe.core.eigensolvers}}

\begin{fulllineitems}
\phantomsection\label{\detokenize{core:pywfe.core.eigensolvers.polynomial}}
\pysigstartsignatures
\pysiglinewithargsret{\sphinxcode{\sphinxupquote{pywfe.core.eigensolvers.}}\sphinxbfcode{\sphinxupquote{polynomial}}}{\emph{\DUrole{n}{DSM}}}{}
\pysigstopsignatures
\sphinxAtStartPar
{[}unfinished{]} Polynomial form of the eigenproblem
\begin{quote}\begin{description}
\sphinxlineitem{Parameters}\begin{description}
\sphinxlineitem{\sphinxstylestrong{DSM}}{[}(N,N) ndarray (float or complex){]}
\sphinxAtStartPar
The dynamic stiffness matristrucaxisym\sphinxhyphen{}x of the system. 
NxN array of type float or complex.

\end{description}

\sphinxlineitem{Returns}\begin{description}
\sphinxlineitem{\sphinxstylestrong{vals}}{[}ndarray{]}
\sphinxAtStartPar
1\sphinxhyphen{}D array of length N type complex.

\sphinxlineitem{\sphinxstylestrong{left\_eigenvectors}}{[}ndarray{]}
\sphinxAtStartPar
NxN array of type float or complex.
Column i is vector corresponding to vals{[}i{]}

\sphinxlineitem{\sphinxstylestrong{right\_eigenvectors}}{[}ndarray{]}
\sphinxAtStartPar
NxN array of type float or complex.
Column i is vector corresponding to vals{[}i{]}

\end{description}

\end{description}\end{quote}

\end{fulllineitems}

\phantomsection\label{\detokenize{core:module-pywfe.core.classify_modes}}\index{module@\spxentry{module}!pywfe.core.classify\_modes@\spxentry{pywfe.core.classify\_modes}}\index{pywfe.core.classify\_modes@\spxentry{pywfe.core.classify\_modes}!module@\spxentry{module}}

\subsection{classify\_modes}
\label{\detokenize{core:classify-modes}}
\sphinxAtStartPar
This module contains the functionality needed to sort eigensolutions of the
WFE method into positive and negative going waves.
\index{classify\_wavemode() (in module pywfe.core.classify\_modes)@\spxentry{classify\_wavemode()}\spxextra{in module pywfe.core.classify\_modes}}

\begin{fulllineitems}
\phantomsection\label{\detokenize{core:pywfe.core.classify_modes.classify_wavemode}}
\pysigstartsignatures
\pysiglinewithargsret{\sphinxcode{\sphinxupquote{pywfe.core.classify\_modes.}}\sphinxbfcode{\sphinxupquote{classify\_wavemode}}}{\emph{\DUrole{n}{f}}, \emph{\DUrole{n}{eigenvalue}}, \emph{\DUrole{n}{eigenvector}}, \emph{\DUrole{n}{threshold}}}{}
\pysigstopsignatures
\sphinxAtStartPar
Identify if a wavemode is positive going or negative going
\begin{quote}\begin{description}
\sphinxlineitem{Parameters}\begin{description}
\sphinxlineitem{\sphinxstylestrong{f}}{[}float{]}
\sphinxAtStartPar
frequency of eigensolution.

\sphinxlineitem{\sphinxstylestrong{eigenvalue}}{[}complex{]}
\sphinxAtStartPar
Eigenvalue to be checked.

\sphinxlineitem{\sphinxstylestrong{eigenvector}}{[}nodarray, complex{]}
\sphinxAtStartPar
Corresponding eigenvector.

\sphinxlineitem{\sphinxstylestrong{threshold}}{[}float{]}
\sphinxAtStartPar
Threshold for classification. How close to unity does an eigenvalue have to be?

\end{description}

\sphinxlineitem{Returns}\begin{description}
\sphinxlineitem{\sphinxstylestrong{direction}}{[}str{]}
\sphinxAtStartPar
\sphinxcode{\sphinxupquote{\textquotesingle{}right\textquotesingle{}}} or \sphinxcode{\sphinxupquote{\textquotesingle{}left\textquotesingle{}}}.

\end{description}

\end{description}\end{quote}

\end{fulllineitems}

\index{sort\_eigensolution() (in module pywfe.core.classify\_modes)@\spxentry{sort\_eigensolution()}\spxextra{in module pywfe.core.classify\_modes}}

\begin{fulllineitems}
\phantomsection\label{\detokenize{core:pywfe.core.classify_modes.sort_eigensolution}}
\pysigstartsignatures
\pysiglinewithargsret{\sphinxcode{\sphinxupquote{pywfe.core.classify\_modes.}}\sphinxbfcode{\sphinxupquote{sort\_eigensolution}}}{\emph{\DUrole{n}{f}}, \emph{\DUrole{n}{eigenvalues}}, \emph{\DUrole{n}{right\_eigenvectors}}, \emph{\DUrole{n}{left\_eigenvectors}}}{}
\pysigstopsignatures
\sphinxAtStartPar
Sort the eigensolution into positive and negative going waves
\begin{quote}\begin{description}
\sphinxlineitem{Parameters}\begin{description}
\sphinxlineitem{\sphinxstylestrong{f}}{[}float{]}
\sphinxAtStartPar
Frequency of eigensolution.

\sphinxlineitem{\sphinxstylestrong{eigenvalues}}{[}ndarray, complex{]}
\sphinxAtStartPar
Eigenvalues solved at this frequency.

\sphinxlineitem{\sphinxstylestrong{right\_eigenvectors}}{[}ndarray, complex{]}
\sphinxAtStartPar
Right eigenvectors solved at this frequency.

\sphinxlineitem{\sphinxstylestrong{left\_eigenvectors}}{[}TYPE{]}
\sphinxAtStartPar
Left eigenvectors solved at this frequency..

\end{description}

\sphinxlineitem{Returns}\begin{description}
\sphinxlineitem{named tuple}
\sphinxAtStartPar
Eigensolution tuple.

\end{description}

\end{description}\end{quote}

\end{fulllineitems}

\phantomsection\label{\detokenize{core:module-pywfe.core.forced_problem}}\index{module@\spxentry{module}!pywfe.core.forced\_problem@\spxentry{pywfe.core.forced\_problem}}\index{pywfe.core.forced\_problem@\spxentry{pywfe.core.forced\_problem}!module@\spxentry{module}}

\subsection{forced\_problem}
\label{\detokenize{core:forced-problem}}
\sphinxAtStartPar
This module contains the functionality needed to apply forces to a WFE model.
\index{calculate\_excited\_amplitudes() (in module pywfe.core.forced\_problem)@\spxentry{calculate\_excited\_amplitudes()}\spxextra{in module pywfe.core.forced\_problem}}

\begin{fulllineitems}
\phantomsection\label{\detokenize{core:pywfe.core.forced_problem.calculate_excited_amplitudes}}
\pysigstartsignatures
\pysiglinewithargsret{\sphinxcode{\sphinxupquote{pywfe.core.forced\_problem.}}\sphinxbfcode{\sphinxupquote{calculate\_excited\_amplitudes}}}{\emph{\DUrole{n}{eigensolution}}, \emph{\DUrole{n}{force}}}{}
\pysigstopsignatures
\sphinxAtStartPar
Calculates the directly excited amplitudes subject to a given force
and modal solution.
\begin{quote}\begin{description}
\sphinxlineitem{Parameters}\begin{description}
\sphinxlineitem{\sphinxstylestrong{eigensolution}}{[}namedtuple{]}
\sphinxAtStartPar
eigensolution.

\sphinxlineitem{\sphinxstylestrong{force}}{[}np.ndarray{]}
\sphinxAtStartPar
force vector.

\end{description}

\sphinxlineitem{Returns}\begin{description}
\sphinxlineitem{\sphinxstylestrong{e\_plus}}{[}np.ndarray{]}
\sphinxAtStartPar
directly excited modal amplitudes (positive).

\sphinxlineitem{\sphinxstylestrong{e\_minus}}{[}np.ndarray{]}
\sphinxAtStartPar
directly excited modal amplitudes (negative).

\end{description}

\end{description}\end{quote}

\end{fulllineitems}

\index{generate\_reflection\_matrices() (in module pywfe.core.forced\_problem)@\spxentry{generate\_reflection\_matrices()}\spxextra{in module pywfe.core.forced\_problem}}

\begin{fulllineitems}
\phantomsection\label{\detokenize{core:pywfe.core.forced_problem.generate_reflection_matrices}}
\pysigstartsignatures
\pysiglinewithargsret{\sphinxcode{\sphinxupquote{pywfe.core.forced\_problem.}}\sphinxbfcode{\sphinxupquote{generate\_reflection\_matrices}}}{\emph{\DUrole{n}{eigensolution}}, \emph{\DUrole{n}{A\_right}}, \emph{\DUrole{n}{B\_right}}, \emph{\DUrole{n}{A\_left}}, \emph{\DUrole{n}{B\_left}}}{}
\pysigstopsignatures
\sphinxAtStartPar
Calculates the reflection matrices from boundary matrices.
\begin{quote}\begin{description}
\sphinxlineitem{Parameters}\begin{description}
\sphinxlineitem{\sphinxstylestrong{eigensolution}}{[}TYPE{]}
\sphinxAtStartPar
DESCRIPTION.

\sphinxlineitem{\sphinxstylestrong{A\_right}}{[}np.ndarray{]}
\sphinxAtStartPar
A matrix on the right boundary.

\sphinxlineitem{\sphinxstylestrong{B\_right}}{[}np.ndarray{]}
\sphinxAtStartPar
B matrix on the right boundary.

\sphinxlineitem{\sphinxstylestrong{A\_left}}{[}np.ndarray{]}
\sphinxAtStartPar
A natrix on the left boundary.

\sphinxlineitem{\sphinxstylestrong{B\_left}}{[}np.ndarray{]}
\sphinxAtStartPar
B matrix on the left boundary.

\end{description}

\sphinxlineitem{Returns}\begin{description}
\sphinxlineitem{\sphinxstylestrong{R\_right}}{[}np.ndarray{]}
\sphinxAtStartPar
Right reflection matrix.

\sphinxlineitem{\sphinxstylestrong{R\_left}}{[}np.ndarray{]}
\sphinxAtStartPar
Left reflection matrix.

\end{description}

\end{description}\end{quote}

\end{fulllineitems}

\index{calculate\_propagated\_amplitudes() (in module pywfe.core.forced\_problem)@\spxentry{calculate\_propagated\_amplitudes()}\spxextra{in module pywfe.core.forced\_problem}}

\begin{fulllineitems}
\phantomsection\label{\detokenize{core:pywfe.core.forced_problem.calculate_propagated_amplitudes}}
\pysigstartsignatures
\pysiglinewithargsret{\sphinxcode{\sphinxupquote{pywfe.core.forced\_problem.}}\sphinxbfcode{\sphinxupquote{calculate\_propagated\_amplitudes}}}{\emph{\DUrole{n}{e\_plus}}, \emph{\DUrole{n}{e\_minus}}, \emph{\DUrole{n}{k\_plus}}, \emph{\DUrole{n}{L}}, \emph{\DUrole{n}{R\_right}}, \emph{\DUrole{n}{R\_left}}, \emph{\DUrole{n}{x\_r}}, \emph{\DUrole{n}{x\_e}\DUrole{o}{=}\DUrole{default_value}{0}}}{}
\pysigstopsignatures
\sphinxAtStartPar
Calculates the ampltiudes of waves after propagation to response point
\begin{quote}\begin{description}
\sphinxlineitem{Parameters}\begin{description}
\sphinxlineitem{\sphinxstylestrong{e\_plus}}{[}np.ndarray{]}
\sphinxAtStartPar
positive directly excited amplitudes.

\sphinxlineitem{\sphinxstylestrong{e\_minus}}{[}np.ndarray{]}
\sphinxAtStartPar
negative directly excited amplitudes.

\sphinxlineitem{\sphinxstylestrong{k\_plus}}{[}np.ndarray{]}
\sphinxAtStartPar
wavenumber array.

\sphinxlineitem{\sphinxstylestrong{L}}{[}float{]}
\sphinxAtStartPar
Length of waveguide.

\sphinxlineitem{\sphinxstylestrong{R\_right}}{[}np.ndarray{]}
\sphinxAtStartPar
Right reflection matrix.

\sphinxlineitem{\sphinxstylestrong{R\_left}}{[}np.ndarray{]}
\sphinxAtStartPar
Left reflection matrix.

\sphinxlineitem{\sphinxstylestrong{x\_r}}{[}float, np.ndarray{]}
\sphinxAtStartPar
Response distance.

\sphinxlineitem{\sphinxstylestrong{x\_e}}{[}float,{]}
\sphinxAtStartPar
Excitation distance. The default is 0.

\end{description}

\sphinxlineitem{Returns}\begin{description}
\sphinxlineitem{\sphinxstylestrong{b\_plus}}{[}np.ndarray{]}
\sphinxAtStartPar
positive propagated amplitudes.

\sphinxlineitem{\sphinxstylestrong{b\_minus}}{[}np.ndarray{]}
\sphinxAtStartPar
negative propagated amplitudes.

\end{description}

\end{description}\end{quote}

\end{fulllineitems}

\index{calculate\_modal\_displacements() (in module pywfe.core.forced\_problem)@\spxentry{calculate\_modal\_displacements()}\spxextra{in module pywfe.core.forced\_problem}}

\begin{fulllineitems}
\phantomsection\label{\detokenize{core:pywfe.core.forced_problem.calculate_modal_displacements}}
\pysigstartsignatures
\pysiglinewithargsret{\sphinxcode{\sphinxupquote{pywfe.core.forced\_problem.}}\sphinxbfcode{\sphinxupquote{calculate\_modal\_displacements}}}{\emph{\DUrole{n}{eigensolution}}, \emph{\DUrole{n}{b\_plus}}, \emph{\DUrole{n}{b\_minus}}}{}
\pysigstopsignatures
\sphinxAtStartPar
Calculates the displacement of each mode (last axis is modal)
\begin{quote}\begin{description}
\sphinxlineitem{Parameters}\begin{description}
\sphinxlineitem{\sphinxstylestrong{eigensolution}}{[}namedtuple{]}
\sphinxAtStartPar
eigensolution.

\sphinxlineitem{\sphinxstylestrong{b\_plus}}{[}np.ndarray{]}
\sphinxAtStartPar
positive propagated amplitudes.

\sphinxlineitem{\sphinxstylestrong{b\_minus}}{[}np.ndarray{]}
\sphinxAtStartPar
negative propagated amplitudes.

\end{description}

\sphinxlineitem{Returns}\begin{description}
\sphinxlineitem{\sphinxstylestrong{q\_j\_plus}}{[}np.ndarray{]}
\sphinxAtStartPar
positive going modal displacements.

\sphinxlineitem{\sphinxstylestrong{q\_j\_minus}}{[}np.ndarray{]}
\sphinxAtStartPar
negative going modal displacements.

\end{description}

\end{description}\end{quote}

\end{fulllineitems}

\index{calculate\_modal\_forces() (in module pywfe.core.forced\_problem)@\spxentry{calculate\_modal\_forces()}\spxextra{in module pywfe.core.forced\_problem}}

\begin{fulllineitems}
\phantomsection\label{\detokenize{core:pywfe.core.forced_problem.calculate_modal_forces}}
\pysigstartsignatures
\pysiglinewithargsret{\sphinxcode{\sphinxupquote{pywfe.core.forced\_problem.}}\sphinxbfcode{\sphinxupquote{calculate\_modal\_forces}}}{\emph{\DUrole{n}{eigensolution}}, \emph{\DUrole{n}{b\_plus}}, \emph{\DUrole{n}{b\_minus}}}{}
\pysigstopsignatures
\sphinxAtStartPar
Calculates the internal forces of each mode (last axis is modal)
\begin{quote}\begin{description}
\sphinxlineitem{Parameters}\begin{description}
\sphinxlineitem{\sphinxstylestrong{eigensolution}}{[}namedtuple{]}
\sphinxAtStartPar
eigensolution.

\sphinxlineitem{\sphinxstylestrong{b\_plus}}{[}np.ndarray{]}
\sphinxAtStartPar
positive propagated amplitudes.

\sphinxlineitem{\sphinxstylestrong{b\_minus}}{[}np.ndarray{]}
\sphinxAtStartPar
negative propagated amplitudes.

\end{description}

\sphinxlineitem{Returns}\begin{description}
\sphinxlineitem{\sphinxstylestrong{f\_j\_plus}}{[}np.ndarray{]}
\sphinxAtStartPar
Positive going modal forces.

\sphinxlineitem{\sphinxstylestrong{f\_j\_minus}}{[}np.ndarray{]}
\sphinxAtStartPar
Negative going modal forces.

\end{description}

\end{description}\end{quote}

\end{fulllineitems}


\sphinxstepscope


\section{utils}
\label{\detokenize{utils:utils}}\label{\detokenize{utils::doc}}
\begin{sphinxShadowBox}
\begin{itemize}
\item {} 
\sphinxAtStartPar
\phantomsection\label{\detokenize{utils:id1}}{\hyperref[\detokenize{utils:io-utils}]{\sphinxcrossref{io\_utils}}}

\item {} 
\sphinxAtStartPar
\phantomsection\label{\detokenize{utils:id2}}{\hyperref[\detokenize{utils:comsol-loader}]{\sphinxcrossref{comsol\_loader}}}

\item {} 
\sphinxAtStartPar
\phantomsection\label{\detokenize{utils:id3}}{\hyperref[\detokenize{utils:frequency-sweep}]{\sphinxcrossref{frequency\_sweep}}}

\item {} 
\sphinxAtStartPar
\phantomsection\label{\detokenize{utils:id4}}{\hyperref[\detokenize{utils:modal-assurance}]{\sphinxcrossref{modal\_assurance}}}

\end{itemize}
\end{sphinxShadowBox}
\phantomsection\label{\detokenize{utils:module-pywfe.utils.io_utils}}\index{module@\spxentry{module}!pywfe.utils.io\_utils@\spxentry{pywfe.utils.io\_utils}}\index{pywfe.utils.io\_utils@\spxentry{pywfe.utils.io\_utils}!module@\spxentry{module}}

\subsection{io\_utils}
\label{\detokenize{utils:io-utils}}
\sphinxAtStartPar
This module contains the functionality needed to save and load pywfe.Model objects
\phantomsection\label{\detokenize{utils:module-pywfe.utils.comsol_loader}}\index{module@\spxentry{module}!pywfe.utils.comsol\_loader@\spxentry{pywfe.utils.comsol\_loader}}\index{pywfe.utils.comsol\_loader@\spxentry{pywfe.utils.comsol\_loader}!module@\spxentry{module}}

\subsection{comsol\_loader}
\label{\detokenize{utils:comsol-loader}}
\sphinxAtStartPar
This module contains the functionality needed to convert COMSOL data
extracted from MATLAB LiveLink into a pywfe.Model class.
\index{load\_comsol() (in module pywfe.utils.comsol\_loader)@\spxentry{load\_comsol()}\spxextra{in module pywfe.utils.comsol\_loader}}

\begin{fulllineitems}
\phantomsection\label{\detokenize{utils:pywfe.utils.comsol_loader.load_comsol}}
\pysigstartsignatures
\pysiglinewithargsret{\sphinxcode{\sphinxupquote{pywfe.utils.comsol\_loader.}}\sphinxbfcode{\sphinxupquote{load\_comsol}}}{\emph{\DUrole{n}{folder}}, \emph{\DUrole{n}{axis}\DUrole{o}{=}\DUrole{default_value}{0}}, \emph{\DUrole{n}{logging\_level}\DUrole{o}{=}\DUrole{default_value}{20}}, \emph{\DUrole{n}{solver}\DUrole{o}{=}\DUrole{default_value}{\textquotesingle{}transfer\_matrix\textquotesingle{}}}}{}
\pysigstopsignatures\begin{quote}\begin{description}
\sphinxlineitem{Parameters}\begin{description}
\sphinxlineitem{\sphinxstylestrong{folder}}{[}string{]}
\sphinxAtStartPar
path to the folder containing the COMSOL LiveLink data.

\sphinxlineitem{\sphinxstylestrong{axis}}{[}int, optional{]}
\sphinxAtStartPar
Waveguide axis. The default is 0.

\sphinxlineitem{\sphinxstylestrong{logging\_level}}{[}int, optional{]}
\sphinxAtStartPar
Logging level. The default is 20 (info).

\end{description}

\sphinxlineitem{Returns}\begin{description}
\sphinxlineitem{\sphinxstylestrong{model}}{[}pywfe.model class{]}
\sphinxAtStartPar
a pywfe model.

\end{description}

\end{description}\end{quote}

\end{fulllineitems}

\index{comsol\_i2j() (in module pywfe.utils.comsol\_loader)@\spxentry{comsol\_i2j()}\spxextra{in module pywfe.utils.comsol\_loader}}

\begin{fulllineitems}
\phantomsection\label{\detokenize{utils:pywfe.utils.comsol_loader.comsol_i2j}}
\pysigstartsignatures
\pysiglinewithargsret{\sphinxcode{\sphinxupquote{pywfe.utils.comsol\_loader.}}\sphinxbfcode{\sphinxupquote{comsol\_i2j}}}{\emph{\DUrole{n}{filename}}, \emph{\DUrole{n}{skiprows}\DUrole{o}{=}\DUrole{default_value}{0}}}{}
\pysigstopsignatures
\sphinxAtStartPar
Converts complex \textquotesingle{}j\textquotesingle{} imaginary unit from COMSOL to python \textquotesingle{}j\textquotesingle{}
\begin{quote}\begin{description}
\sphinxlineitem{Parameters}\begin{description}
\sphinxlineitem{\sphinxstylestrong{filename}}{[}string{]}
\sphinxAtStartPar
filename to convert.

\sphinxlineitem{\sphinxstylestrong{skiprows}}{[}int, optional{]}
\sphinxAtStartPar
see numpy loadtxt. The default is 1.

\end{description}

\sphinxlineitem{Returns}\begin{description}
\sphinxlineitem{None.}
\end{description}

\end{description}\end{quote}

\end{fulllineitems}

\phantomsection\label{\detokenize{utils:module-pywfe.utils.frequency_sweep}}\index{module@\spxentry{module}!pywfe.utils.frequency\_sweep@\spxentry{pywfe.utils.frequency\_sweep}}\index{pywfe.utils.frequency\_sweep@\spxentry{pywfe.utils.frequency\_sweep}!module@\spxentry{module}}

\subsection{frequency\_sweep}
\label{\detokenize{utils:frequency-sweep}}
\sphinxAtStartPar
This module contains the  fucntion for calculating various quantities over
an array of frequencies for a pywfe.Model object.
\index{frequency\_sweep() (in module pywfe.utils.frequency\_sweep)@\spxentry{frequency\_sweep()}\spxextra{in module pywfe.utils.frequency\_sweep}}

\begin{fulllineitems}
\phantomsection\label{\detokenize{utils:pywfe.utils.frequency_sweep.frequency_sweep}}
\pysigstartsignatures
\pysiglinewithargsret{\sphinxcode{\sphinxupquote{pywfe.utils.frequency\_sweep.}}\sphinxbfcode{\sphinxupquote{frequency\_sweep}}}{\emph{\DUrole{n}{model}}, \emph{\DUrole{n}{f\_arr}}, \emph{\DUrole{n}{quantities}}, \emph{\DUrole{n}{x\_r}\DUrole{o}{=}\DUrole{default_value}{0}}, \emph{\DUrole{n}{mac}\DUrole{o}{=}\DUrole{default_value}{False}}, \emph{\DUrole{n}{dofs}\DUrole{o}{=}\DUrole{default_value}{\textquotesingle{}all\textquotesingle{}}}}{}
\pysigstopsignatures
\sphinxAtStartPar
Perform a sweep over frequency array, extracting specified quatities
at each step. Modal assurance criterion can be used to track modes through
frequency by modeshape similarity
\begin{quote}\begin{description}
\sphinxlineitem{Parameters}\begin{description}
\sphinxlineitem{\sphinxstylestrong{model}}{[}pywfe.Model{]}
\sphinxAtStartPar
The model to perform the sweep with.

\sphinxlineitem{\sphinxstylestrong{f\_arr}}{[}np.ndarray float{]}
\sphinxAtStartPar
frequency array.

\sphinxlineitem{\sphinxstylestrong{quantities}}{[}list of str type{]}
\sphinxAtStartPar
a list of strings specifying the quantities to be calculated.
These are:
\sphinxhyphen{} phi\_plus: the (positive going) eigenvectors
\sphinxhyphen{} excited\_amplitudes: see \sphinxtitleref{pywfe.Model.excited\_amplitudes}
\sphinxhyphen{} propagated\_amplitudes: see \sphinxtitleref{pywfe.Model.propagated\_amplitudes}
\sphinxhyphen{} modal\_displacements: see \sphinxtitleref{pywfe.Model.modal\_displacements}
\sphinxhyphen{} wavenumbers: see \sphinxtitleref{pywfe.Model.wavenumbers}
\sphinxhyphen{} displacements: see \sphinxtitleref{pywfe.Model.displacements}
\sphinxhyphen{} forces: see \sphinxtitleref{pywfe.Model.forces}

\sphinxlineitem{\sphinxstylestrong{x\_r}}{[}float, np.ndarray, optional{]}
\sphinxAtStartPar
response distance(s). The default is 0.

\sphinxlineitem{\sphinxstylestrong{mac}}{[}bool, optional{]}
\sphinxAtStartPar
Use the modal assurance criterion to sort waves. The default is False.

\sphinxlineitem{\sphinxstylestrong{dofs}}{[}dofs, optional{]}
\sphinxAtStartPar
The selected degrees of freedom. See \sphinxtitleref{pywfe.Model.dofs\_to\_inds}.
The default is \textquotesingle{}all\textquotesingle{}.

\end{description}

\sphinxlineitem{Returns}\begin{description}
\sphinxlineitem{\sphinxstylestrong{output}}{[}dict{]}
\sphinxAtStartPar
Dictionary of outputs for specified quantities.

\end{description}

\end{description}\end{quote}

\end{fulllineitems}

\phantomsection\label{\detokenize{utils:module-pywfe.utils.modal_assurance}}\index{module@\spxentry{module}!pywfe.utils.modal\_assurance@\spxentry{pywfe.utils.modal\_assurance}}\index{pywfe.utils.modal\_assurance@\spxentry{pywfe.utils.modal\_assurance}!module@\spxentry{module}}

\subsection{modal\_assurance}
\label{\detokenize{utils:modal-assurance}}
\sphinxAtStartPar
This module contains functions for sorting frequency sweept data by mode index
\index{mac\_matrix() (in module pywfe.utils.modal\_assurance)@\spxentry{mac\_matrix()}\spxextra{in module pywfe.utils.modal\_assurance}}

\begin{fulllineitems}
\phantomsection\label{\detokenize{utils:pywfe.utils.modal_assurance.mac_matrix}}
\pysigstartsignatures
\pysiglinewithargsret{\sphinxcode{\sphinxupquote{pywfe.utils.modal\_assurance.}}\sphinxbfcode{\sphinxupquote{mac\_matrix}}}{\emph{\DUrole{n}{modes\_prev}}, \emph{\DUrole{n}{modes\_next}}}{}
\pysigstopsignatures
\sphinxAtStartPar
Compute the Modal Assurance Criterion (MAC) matrix.

\end{fulllineitems}

\phantomsection\label{\detokenize{utils:module-pywfe.utils.shaker}}\index{module@\spxentry{module}!pywfe.utils.shaker@\spxentry{pywfe.utils.shaker}}\index{pywfe.utils.shaker@\spxentry{pywfe.utils.shaker}!module@\spxentry{module}}
\sphinxAtStartPar
Created on Tue Aug 22 11:24:28 2023

\sphinxAtStartPar
@author: Austen

\sphinxstepscope


\section{Examples}
\label{\detokenize{examples/index:examples}}\label{\detokenize{examples/index:examples-index}}\label{\detokenize{examples/index::doc}}
\sphinxAtStartPar
Here you can find examples that demonstrate how to use the \sphinxtitleref{pywfe} package.

\sphinxstepscope


\subsection{Analytical Beam Example}
\label{\detokenize{examples/analytical_beam:analytical-beam-example}}\label{\detokenize{examples/analytical_beam::doc}}
\sphinxAtStartPar
In this example, we’ll go through the process of setting up a model of an \sphinxhref{https://en.wikipedia.org/wiki/Euler\%E2\%80\%93Bernoulli\_beam\_theory}{Euler\sphinxhyphen{}Bernoulli} beam using the \sphinxcode{\sphinxupquote{pywfe}} package.


\subsubsection{Introduction}
\label{\detokenize{examples/analytical_beam:introduction}}
\noindent{\hspace*{\fill}\sphinxincludegraphics{{beam_element}.png}\hspace*{\fill}}

\sphinxAtStartPar
An Euler\sphinxhyphen{}Bernoulli beam can be described with a finite element approximation giving the mass and stiffness matrices:
\begin{equation*}
\begin{split}\mathbf{M}=\frac{\rho A l}{420}\left[\begin{array}{cccc}
156 & 22 l & 54 & -13 l \\
22 l & 4 l^2 & 13 l & -3 l^2 \\
54 & 13 l & 156 & -22 l \\
-13 l & -3 l^2 & -22 l & 4 l^2
\end{array}\right] \quad\mathbf{K}=\frac{E I}{l^3}\left[\begin{array}{cccc}
12 & 6 l & -12 & 6 l \\
6 l & 4 l^2 & -6 l & 2 l^2 \\
-12 & -6 l & 12 & -6 l \\
6 l & 2 l^2 & -6 l & 4 l^2
\end{array}\right]\end{split}
\end{equation*}
\sphinxAtStartPar
For a beam segment of length \(l\), cross\sphinxhyphen{}sectional area \(A\) made from a material with Young’s modulus and density \(E, \rho\), and second moment of area \(I\).
These matrices relate the displacement/rotation vector \([w_1, \theta_1, w_2, \theta_2]^T\) with the force/moment vector \([F_1, M_1, f_2, F_2]^T\) by
\begin{equation*}
\begin{split}\begin{bmatrix}
w_1\\
\theta_1\\
w_2\\
\theta_2
\end{bmatrix} \left(\mathbf{K} - \omega^2 \mathbf{M} \right) = \begin{bmatrix} F_1\\
M_1\\
F_2\\
M_2
\end{bmatrix}\end{split}
\end{equation*}
\sphinxAtStartPar
The FE model only has two nodes with two degrees of freedom each. The analytical formulation of an infinite beam has well known solutions.
The dispersion relation for transverse waves is
\begin{equation*}
\begin{split}k = \sqrt{ \frac{\omega}{a} }\end{split}
\end{equation*}
\sphinxAtStartPar
The transfer mobility is subject to a transverse point force at \(x=0\) is
\begin{equation*}
\begin{split}v(x, \omega)=-\frac{\omega}{4 E I k^3}\left(i e^{-k x}-e^{-i k x}\right)\end{split}
\end{equation*}

\subsubsection{Creating pywfe Model of Beam}
\label{\detokenize{examples/analytical_beam:creating-pywfe-model-of-beam}}
\sphinxAtStartPar
To begin with we define the system parameters

\begin{sphinxVerbatim}[commandchars=\\\{\}]
\PYG{k+kn}{import} \PYG{n+nn}{numpy} \PYG{k}{as} \PYG{n+nn}{np}
\PYG{k+kn}{import} \PYG{n+nn}{pywfe}
\PYG{k+kn}{import} \PYG{n+nn}{matplotlib}\PYG{n+nn}{.}\PYG{n+nn}{pyplot} \PYG{k}{as} \PYG{n+nn}{plots}

\PYG{n}{E} \PYG{o}{=} \PYG{l+m+mf}{2.1e11}  \PYG{c+c1}{\PYGZsh{} young mod}
\PYG{n}{rho} \PYG{o}{=} \PYG{l+m+mi}{7850}  \PYG{c+c1}{\PYGZsh{} density}
\PYG{n}{h} \PYG{o}{=} \PYG{l+m+mf}{0.1}  \PYG{c+c1}{\PYGZsh{} bean cross section side length length}
\PYG{n}{A} \PYG{o}{=} \PYG{n}{h}\PYG{o}{*}\PYG{o}{*}\PYG{l+m+mi}{2}  \PYG{c+c1}{\PYGZsh{} beam cross sectional area}
\PYG{n}{I} \PYG{o}{=} \PYG{n}{h}\PYG{o}{*}\PYG{o}{*}\PYG{l+m+mi}{4} \PYG{o}{/} \PYG{l+m+mi}{12}  \PYG{c+c1}{\PYGZsh{} second moment of area}

\PYG{n}{a} \PYG{o}{=} \PYG{n}{np}\PYG{o}{.}\PYG{n}{sqrt}\PYG{p}{(}\PYG{n}{E}\PYG{o}{*}\PYG{n}{I}\PYG{o}{/}\PYG{p}{(}\PYG{n}{rho}\PYG{o}{*}\PYG{n}{A}\PYG{p}{)}\PYG{p}{)}  \PYG{c+c1}{\PYGZsh{} factor in dispersion relation}
\end{sphinxVerbatim}

\sphinxAtStartPar
and define the known solutions for the analytical dispersion relation and transfer mobility

\begin{sphinxVerbatim}[commandchars=\\\{\}]
\PYG{k}{def} \PYG{n+nf}{euler\PYGZus{}wavenumber}\PYG{p}{(}\PYG{n}{f}\PYG{p}{)}\PYG{p}{:}
    \PYG{c+c1}{\PYGZsh{} wavenumber of euler bernoulli beam}
    \PYG{k}{return} \PYG{n}{np}\PYG{o}{.}\PYG{n}{sqrt}\PYG{p}{(}\PYG{l+m+mi}{2}\PYG{o}{*}\PYG{n}{np}\PYG{o}{.}\PYG{n}{pi}\PYG{o}{*}\PYG{n}{f}\PYG{o}{/}\PYG{n}{a}\PYG{p}{)}


\PYG{k}{def} \PYG{n+nf}{transfer\PYGZus{}velocity}\PYG{p}{(}\PYG{n}{f}\PYG{p}{,} \PYG{n}{x}\PYG{p}{)}\PYG{p}{:}
    \PYG{c+c1}{\PYGZsh{} transfer velocity for beam x \PYGZgt{} 0}
    \PYG{n}{k} \PYG{o}{=} \PYG{n}{euler\PYGZus{}wavenumber}\PYG{p}{(}\PYG{n}{f}\PYG{p}{)}
    \PYG{n}{omega} \PYG{o}{=} \PYG{l+m+mi}{2}\PYG{o}{*}\PYG{n}{np}\PYG{o}{.}\PYG{n}{pi}\PYG{o}{*}\PYG{n}{f}

    \PYG{k}{return} \PYG{o}{\PYGZhy{}}\PYG{n}{omega}\PYG{o}{/}\PYG{p}{(}\PYG{l+m+mi}{4}\PYG{o}{*}\PYG{n}{E}\PYG{o}{*}\PYG{n}{I}\PYG{o}{*}\PYG{n}{k}\PYG{o}{*}\PYG{o}{*}\PYG{l+m+mi}{3}\PYG{p}{)} \PYG{o}{*} \PYG{p}{(}\PYG{l+m+mi}{1}\PYG{n}{j}\PYG{o}{*}\PYG{n}{np}\PYG{o}{.}\PYG{n}{exp}\PYG{p}{(}\PYG{o}{\PYGZhy{}}\PYG{n}{k}\PYG{o}{*}\PYG{n}{x}\PYG{p}{)} \PYG{o}{\PYGZhy{}} \PYG{n}{np}\PYG{o}{.}\PYG{n}{exp}\PYG{p}{(}\PYG{o}{\PYGZhy{}}\PYG{l+m+mi}{1}\PYG{n}{j}\PYG{o}{*}\PYG{n}{k}\PYG{o}{*}\PYG{n}{x}\PYG{p}{)}\PYG{p}{)}
\end{sphinxVerbatim}

\sphinxAtStartPar
For the FE discretisation, the beam length must be significantly shorter than the minimum wavelength. We define maximum frequency and find the maximum wavenumber analytically to set the beam length for WFE modelling.

\begin{sphinxVerbatim}[commandchars=\\\{\}]
\PYG{n}{f\PYGZus{}max} \PYG{o}{=} \PYG{l+m+mf}{1e3}  \PYG{c+c1}{\PYGZsh{} maximum frequency}
\PYG{n}{lambda\PYGZus{}min} \PYG{o}{=} \PYG{l+m+mi}{2}\PYG{o}{*}\PYG{n}{np}\PYG{o}{.}\PYG{n}{pi}\PYG{o}{/}\PYG{n}{euler\PYGZus{}wavenumber}\PYG{p}{(}\PYG{n}{f\PYGZus{}max}\PYG{p}{)}  \PYG{c+c1}{\PYGZsh{} mimimum wavelength}
\PYG{n}{l\PYGZus{}max} \PYG{o}{=} \PYG{n}{lambda\PYGZus{}min} \PYG{o}{/} \PYG{l+m+mi}{10}  \PYG{c+c1}{\PYGZsh{} unit cell length max \PYGZhy{} 10 unit cells per wavelength}

\PYG{n}{l} \PYG{o}{=} \PYG{n}{np}\PYG{o}{.}\PYG{n}{round}\PYG{p}{(}\PYG{n}{l\PYGZus{}max}\PYG{p}{,} \PYG{n}{decimals}\PYG{o}{=}\PYG{l+m+mi}{1}\PYG{p}{)}  \PYG{c+c1}{\PYGZsh{} rounded unit cell length chosen}
\end{sphinxVerbatim}

\sphinxAtStartPar
Now the mass and stiffness matrices can be defined

\begin{sphinxVerbatim}[commandchars=\\\{\}]
\PYG{c+c1}{\PYGZsh{} stiffness matrix}
\PYG{n}{K} \PYG{o}{=} \PYG{n}{E}\PYG{o}{*}\PYG{n}{I}\PYG{o}{/}\PYG{p}{(}\PYG{n}{l}\PYG{o}{*}\PYG{o}{*}\PYG{l+m+mi}{3}\PYG{p}{)} \PYG{o}{*} \PYG{n}{np}\PYG{o}{.}\PYG{n}{array}\PYG{p}{(}\PYG{p}{[}

    \PYG{p}{[}\PYG{l+m+mi}{12}\PYG{p}{,}    \PYG{l+m+mi}{6}\PYG{o}{*}\PYG{n}{l}\PYG{p}{,}    \PYG{o}{\PYGZhy{}}\PYG{l+m+mi}{12}\PYG{p}{,}   \PYG{l+m+mi}{6}\PYG{o}{*}\PYG{n}{l}\PYG{p}{]}\PYG{p}{,}
    \PYG{p}{[}\PYG{l+m+mi}{6}\PYG{o}{*}\PYG{n}{l}\PYG{p}{,} \PYG{l+m+mi}{4}\PYG{o}{*}\PYG{n}{l}\PYG{o}{*}\PYG{o}{*}\PYG{l+m+mi}{2}\PYG{p}{,} \PYG{o}{\PYGZhy{}}\PYG{l+m+mi}{6}\PYG{o}{*}\PYG{n}{l}\PYG{p}{,} \PYG{l+m+mi}{2}\PYG{o}{*}\PYG{n}{l}\PYG{o}{*}\PYG{o}{*}\PYG{l+m+mi}{2}\PYG{p}{]}\PYG{p}{,}
    \PYG{p}{[}\PYG{o}{\PYGZhy{}}\PYG{l+m+mi}{12}\PYG{p}{,}   \PYG{o}{\PYGZhy{}}\PYG{l+m+mi}{6}\PYG{o}{*}\PYG{n}{l}\PYG{p}{,}    \PYG{l+m+mi}{12}\PYG{p}{,}  \PYG{o}{\PYGZhy{}}\PYG{l+m+mi}{6}\PYG{o}{*}\PYG{n}{l}\PYG{p}{]}\PYG{p}{,}
    \PYG{p}{[}\PYG{l+m+mi}{6}\PYG{o}{*}\PYG{n}{l}\PYG{p}{,} \PYG{l+m+mi}{2}\PYG{o}{*}\PYG{n}{l}\PYG{o}{*}\PYG{o}{*}\PYG{l+m+mi}{2}\PYG{p}{,} \PYG{o}{\PYGZhy{}}\PYG{l+m+mi}{6}\PYG{o}{*}\PYG{n}{l}\PYG{p}{,} \PYG{l+m+mi}{4}\PYG{o}{*}\PYG{n}{l}\PYG{o}{*}\PYG{o}{*}\PYG{l+m+mi}{2}\PYG{p}{]}

\PYG{p}{]}\PYG{p}{)}

\PYG{c+c1}{\PYGZsh{} mass matrix}
\PYG{n}{M} \PYG{o}{=} \PYG{n}{rho}\PYG{o}{*}\PYG{n}{A}\PYG{o}{*}\PYG{n}{l}\PYG{o}{/}\PYG{l+m+mi}{420} \PYG{o}{*} \PYG{n}{np}\PYG{o}{.}\PYG{n}{array}\PYG{p}{(}\PYG{p}{[}

    \PYG{p}{[}\PYG{l+m+mi}{156}\PYG{p}{,}   \PYG{l+m+mi}{22}\PYG{o}{*}\PYG{n}{l}\PYG{p}{,}      \PYG{l+m+mi}{54}\PYG{p}{,}    \PYG{o}{\PYGZhy{}}\PYG{l+m+mi}{13}\PYG{o}{*}\PYG{n}{l}\PYG{p}{]}\PYG{p}{,}
    \PYG{p}{[}\PYG{l+m+mi}{22}\PYG{o}{*}\PYG{n}{l}\PYG{p}{,}  \PYG{l+m+mi}{4}\PYG{o}{*}\PYG{n}{l}\PYG{o}{*}\PYG{o}{*}\PYG{l+m+mi}{2}\PYG{p}{,}  \PYG{l+m+mi}{13}\PYG{o}{*}\PYG{n}{l}\PYG{p}{,}  \PYG{o}{\PYGZhy{}}\PYG{l+m+mi}{3}\PYG{o}{*}\PYG{n}{l}\PYG{o}{*}\PYG{o}{*}\PYG{l+m+mi}{2}\PYG{p}{]}\PYG{p}{,}
    \PYG{p}{[}\PYG{l+m+mi}{54}\PYG{p}{,}    \PYG{l+m+mi}{13}\PYG{o}{*}\PYG{n}{l}\PYG{p}{,}     \PYG{l+m+mi}{156}\PYG{p}{,}    \PYG{o}{\PYGZhy{}}\PYG{l+m+mi}{22}\PYG{o}{*}\PYG{n}{l}\PYG{p}{]}\PYG{p}{,}
    \PYG{p}{[}\PYG{o}{\PYGZhy{}}\PYG{l+m+mi}{13}\PYG{o}{*}\PYG{n}{l}\PYG{p}{,} \PYG{o}{\PYGZhy{}}\PYG{l+m+mi}{3}\PYG{o}{*}\PYG{n}{l}\PYG{o}{*}\PYG{o}{*}\PYG{l+m+mi}{2}\PYG{p}{,}  \PYG{o}{\PYGZhy{}}\PYG{l+m+mi}{22}\PYG{o}{*}\PYG{n}{l}\PYG{p}{,} \PYG{l+m+mi}{4}\PYG{o}{*}\PYG{n}{l}\PYG{o}{*}\PYG{o}{*}\PYG{l+m+mi}{2}\PYG{p}{]}

\PYG{p}{]}\PYG{p}{)}
\end{sphinxVerbatim}

\sphinxAtStartPar
These, along with the ‘mesh’ information are all that are needed to create the \sphinxtitleref{pywfe.Model} object. The mesh information is given with a dictionary with three keys \sphinxtitleref{node}, \sphinxtitleref{fieldvar} and \sphinxtitleref{coord}.
These specify the node number, field variable, and coordinates in 1\sphinxhyphen{}3D of each degree of freedom in the model. The beam has 4 degrees of freedom, ordered as in the displacement vectors. Thus we define the \sphinxtitleref{dof} dictionary

\begin{sphinxVerbatim}[commandchars=\\\{\}]
\PYG{n}{dof} \PYG{o}{=} \PYG{p}{\PYGZob{}}\PYG{l+s+s1}{\PYGZsq{}}\PYG{l+s+s1}{node}\PYG{l+s+s1}{\PYGZsq{}}\PYG{p}{:} \PYG{p}{[}\PYG{l+m+mi}{0}\PYG{p}{,} \PYG{l+m+mi}{0}\PYG{p}{,} \PYG{l+m+mi}{1}\PYG{p}{,} \PYG{l+m+mi}{1}\PYG{p}{]}\PYG{p}{,}
    \PYG{l+s+s1}{\PYGZsq{}}\PYG{l+s+s1}{fieldvar}\PYG{l+s+s1}{\PYGZsq{}}\PYG{p}{:} \PYG{p}{[}\PYG{l+s+s1}{\PYGZsq{}}\PYG{l+s+s1}{w}\PYG{l+s+s1}{\PYGZsq{}}\PYG{p}{,} \PYG{l+s+s1}{\PYGZsq{}}\PYG{l+s+s1}{phi}\PYG{l+s+s1}{\PYGZsq{}}\PYG{p}{]}\PYG{o}{*}\PYG{l+m+mi}{2}\PYG{p}{,}
    \PYG{l+s+s1}{\PYGZsq{}}\PYG{l+s+s1}{coord}\PYG{l+s+s1}{\PYGZsq{}}\PYG{p}{:} \PYG{p}{[}
                \PYG{p}{[}\PYG{l+m+mi}{0}\PYG{p}{,} \PYG{l+m+mi}{0}\PYG{p}{,} \PYG{n}{l}\PYG{p}{,} \PYG{n}{l}\PYG{p}{]}\PYG{p}{,}
                \PYG{p}{[}\PYG{l+m+mi}{0}\PYG{p}{,} \PYG{l+m+mi}{0}\PYG{p}{,} \PYG{l+m+mi}{0}\PYG{p}{,} \PYG{l+m+mi}{0}\PYG{p}{]}
\PYG{p}{]}
\PYG{p}{\PYGZcb{}}
\end{sphinxVerbatim}

\sphinxAtStartPar
which describes the two nodes, the field quantities \sphinxtitleref{w}, \sphinxtitleref{phi} (repeated on each node), and the coordinates of each degree of freedom.
The coordinates are given in \sphinxtitleref{x} and \sphinxtitleref{y} with two lists for demonstrative purposes. Only the first is required for this 1D model.

\sphinxAtStartPar
The pywfe.Model object can now be created

\begin{sphinxVerbatim}[commandchars=\\\{\}]
\PYG{n}{beam\PYGZus{}model} \PYG{o}{=} \PYG{n}{pywfe}\PYG{o}{.}\PYG{n}{Model}\PYG{p}{(}\PYG{n}{K}\PYG{p}{,} \PYG{n}{M}\PYG{p}{,} \PYG{n}{dof}\PYG{p}{)}
\end{sphinxVerbatim}

\sphinxAtStartPar
At this point, you might want to check the model with {\hyperref[\detokenize{model:pywfe.Model.see}]{\sphinxcrossref{\sphinxcode{\sphinxupquote{pywfe.Model.see()}}}}}, which creates an interactive matplotlib view of the nodes in the mesh.
In this case however there is only one node to look at.


\subsubsection{Usage}
\label{\detokenize{examples/analytical_beam:usage}}

\paragraph{Free Waves}
\label{\detokenize{examples/analytical_beam:free-waves}}
\sphinxAtStartPar
Firstly let’s check the dispersion relation with the analytical solution

\begin{sphinxVerbatim}[commandchars=\\\{\}]
\PYG{c+c1}{\PYGZsh{}create frequency array}
\PYG{n}{f\PYGZus{}arr} \PYG{o}{=} \PYG{n}{np}\PYG{o}{.}\PYG{n}{linspace}\PYG{p}{(}\PYG{l+m+mi}{10}\PYG{p}{,} \PYG{n}{f\PYGZus{}max}\PYG{p}{,} \PYG{l+m+mi}{100}\PYG{p}{)}

\PYG{c+c1}{\PYGZsh{} calculate the wfe wavenumbers}
\PYG{n}{k\PYGZus{}wfe} \PYG{o}{=} \PYG{n}{beam\PYGZus{}model}\PYG{o}{.}\PYG{n}{dispersion\PYGZus{}relation}\PYG{p}{(}\PYG{n}{f\PYGZus{}arr}\PYG{p}{)}

\PYG{n}{plt}\PYG{o}{.}\PYG{n}{plot}\PYG{p}{(}\PYG{n}{f\PYGZus{}arr}\PYG{p}{,} \PYG{n}{euler\PYGZus{}wavenumber}\PYG{p}{(}\PYG{n}{f\PYGZus{}arr}\PYG{p}{)}\PYG{p}{,} \PYG{l+s+s1}{\PYGZsq{}}\PYG{l+s+s1}{.}\PYG{l+s+s1}{\PYGZsq{}}\PYG{p}{,} \PYG{n}{color}\PYG{o}{=}\PYG{l+s+s1}{\PYGZsq{}}\PYG{l+s+s1}{red}\PYG{l+s+s1}{\PYGZsq{}}\PYG{p}{,} \PYG{n}{label}\PYG{o}{=}\PYG{l+s+s1}{\PYGZsq{}}\PYG{l+s+s1}{analytical}\PYG{l+s+s1}{\PYGZsq{}}\PYG{p}{)}
\PYG{n}{plt}\PYG{o}{.}\PYG{n}{plot}\PYG{p}{(}\PYG{n}{f\PYGZus{}arr}\PYG{p}{,} \PYG{n}{k\PYGZus{}wfe}\PYG{p}{,} \PYG{n}{color}\PYG{o}{=}\PYG{l+s+s1}{\PYGZsq{}}\PYG{l+s+s1}{black}\PYG{l+s+s1}{\PYGZsq{}}\PYG{p}{)}

\PYG{n}{plt}\PYG{o}{.}\PYG{n}{legend}\PYG{p}{(}\PYG{n}{loc}\PYG{o}{=}\PYG{l+s+s1}{\PYGZsq{}}\PYG{l+s+s1}{best}\PYG{l+s+s1}{\PYGZsq{}}\PYG{p}{)}
\PYG{n}{plt}\PYG{o}{.}\PYG{n}{xlabel}\PYG{p}{(}\PYG{l+s+s2}{\PYGZdq{}}\PYG{l+s+s2}{frequency (Hz)}\PYG{l+s+s2}{\PYGZdq{}}\PYG{p}{)}
\PYG{n}{plt}\PYG{o}{.}\PYG{n}{ylabel}\PYG{p}{(}\PYG{l+s+s2}{\PYGZdq{}}\PYG{l+s+s2}{wavenumber (1/m)}\PYG{l+s+s2}{\PYGZdq{}}\PYG{p}{)}
\end{sphinxVerbatim}

\noindent{\hspace*{\fill}\sphinxincludegraphics{{beam_dispersion_relation}.png}\hspace*{\fill}}


\paragraph{Forcing}
\label{\detokenize{examples/analytical_beam:forcing}}
\sphinxAtStartPar
Forces can be added to degrees of freedom by changing elements of the \sphinxtitleref{Model.force} array. We compare the mobility in the WFE model with the known solution

\begin{sphinxVerbatim}[commandchars=\\\{\}]
\PYG{n}{beam\PYGZus{}model}\PYG{o}{.}\PYG{n}{force}\PYG{p}{[}\PYG{l+m+mi}{0}\PYG{p}{]} \PYG{o}{=} \PYG{l+m+mi}{1}

\PYG{n}{x\PYGZus{}r} \PYG{o}{=} \PYG{l+m+mi}{0}

\PYG{n}{w} \PYG{o}{=} \PYG{n}{beam\PYGZus{}model}\PYG{o}{.}\PYG{n}{transfer\PYGZus{}function}\PYG{p}{(}\PYG{n}{f\PYGZus{}arr}\PYG{p}{,} \PYG{n}{x\PYGZus{}r}\PYG{o}{=}\PYG{n}{x\PYGZus{}r}\PYG{p}{,} \PYG{n}{dofs}\PYG{o}{=}\PYG{p}{[}\PYG{l+m+mi}{0}\PYG{p}{]}\PYG{p}{,} \PYG{n}{derivative}\PYG{o}{=}\PYG{l+m+mi}{1}\PYG{p}{)}

\PYG{n}{plt}\PYG{o}{.}\PYG{n}{semilogy}\PYG{p}{(}\PYG{n}{f\PYGZus{}arr}\PYG{p}{,} \PYG{n+nb}{abs}\PYG{p}{(}\PYG{n}{transfer\PYGZus{}velocity}\PYG{p}{(}\PYG{n}{f\PYGZus{}arr}\PYG{p}{,} \PYG{n}{x\PYGZus{}r}\PYG{p}{)}\PYG{p}{)}\PYG{p}{,} \PYG{l+s+s1}{\PYGZsq{}}\PYG{l+s+s1}{.}\PYG{l+s+s1}{\PYGZsq{}}\PYG{p}{,} \PYG{n}{color}\PYG{o}{=}\PYG{l+s+s1}{\PYGZsq{}}\PYG{l+s+s1}{red}\PYG{l+s+s1}{\PYGZsq{}}\PYG{p}{,} \PYG{n}{label}\PYG{o}{=}\PYG{l+s+s1}{\PYGZsq{}}\PYG{l+s+s1}{analytical}\PYG{l+s+s1}{\PYGZsq{}}\PYG{p}{)}
\PYG{n}{plt}\PYG{o}{.}\PYG{n}{semilogy}\PYG{p}{(}\PYG{n}{f\PYGZus{}arr}\PYG{p}{,} \PYG{n+nb}{abs}\PYG{p}{(}\PYG{n}{w}\PYG{p}{)}\PYG{p}{,} \PYG{n}{color}\PYG{o}{=}\PYG{l+s+s1}{\PYGZsq{}}\PYG{l+s+s1}{black}\PYG{l+s+s1}{\PYGZsq{}}\PYG{p}{,} \PYG{n}{label}\PYG{o}{=}\PYG{l+s+s1}{\PYGZsq{}}\PYG{l+s+s1}{WFE}\PYG{l+s+s1}{\PYGZsq{}}\PYG{p}{)}

\PYG{n}{plt}\PYG{o}{.}\PYG{n}{legend}\PYG{p}{(}\PYG{n}{loc}\PYG{o}{=}\PYG{l+s+s1}{\PYGZsq{}}\PYG{l+s+s1}{best}\PYG{l+s+s1}{\PYGZsq{}}\PYG{p}{)}
\PYG{n}{plt}\PYG{o}{.}\PYG{n}{xlabel}\PYG{p}{(}\PYG{l+s+s2}{\PYGZdq{}}\PYG{l+s+s2}{frequency (Hz)}\PYG{l+s+s2}{\PYGZdq{}}\PYG{p}{)}
\PYG{n}{plt}\PYG{o}{.}\PYG{n}{ylabel}\PYG{p}{(}\PYG{l+s+s2}{\PYGZdq{}}\PYG{l+s+s2}{abs(mobility) (m/(Ns)}\PYG{l+s+s2}{\PYGZdq{}}\PYG{p}{)}
\end{sphinxVerbatim}

\sphinxAtStartPar
The \sphinxtitleref{transfer\_function} method calculates the response over all frequencies at the response distance \sphinxtitleref{x\_r}. The response distance can also be a list or array, in which case a higher dimensional array will be returned.
The \sphinxtitleref{dofs} keyword argument specifies for which degrees of freedom the output should be returned. In this case we want the same dof as the one we’re forcing. The \sphinxtitleref{derivative} keyword argument applies n derivatives in the
frequency domain, i.e a multiplication of the displacement by \(i \omega\). So the output of the method call is the transverse velocity at x=0 for a transverse unit point force. This is the mobility of the beam and is compared
with the analytical solution.

\noindent{\hspace*{\fill}\sphinxincludegraphics{{beam_transfer_mobility}.png}\hspace*{\fill}}

\sphinxAtStartPar
See {\hyperref[\detokenize{model:pywfe.Model.transfer_function}]{\sphinxcrossref{\sphinxcode{\sphinxupquote{pywfe.Model.transfer\_function()}}}}} for more information


\paragraph{More Functionality}
\label{\detokenize{examples/analytical_beam:more-functionality}}
\sphinxAtStartPar
For more functionality see {\hyperref[\detokenize{model:pywfe.Model}]{\sphinxcrossref{\sphinxcode{\sphinxupquote{pywfe.Model}}}}}


\chapter{Indices and tables}
\label{\detokenize{index:indices-and-tables}}\begin{itemize}
\item {} 
\sphinxAtStartPar
\DUrole{xref,std,std-ref}{genindex}

\item {} 
\sphinxAtStartPar
\DUrole{xref,std,std-ref}{modindex}

\item {} 
\sphinxAtStartPar
\DUrole{xref,std,std-ref}{search}

\end{itemize}


\renewcommand{\indexname}{Python Module Index}
\begin{sphinxtheindex}
\let\bigletter\sphinxstyleindexlettergroup
\bigletter{p}
\item\relax\sphinxstyleindexentry{pywfe}\sphinxstyleindexpageref{pywfe:\detokenize{module-pywfe}}
\item\relax\sphinxstyleindexentry{pywfe.core.classify\_modes}\sphinxstyleindexpageref{core:\detokenize{module-pywfe.core.classify_modes}}
\item\relax\sphinxstyleindexentry{pywfe.core.eigensolvers}\sphinxstyleindexpageref{core:\detokenize{module-pywfe.core.eigensolvers}}
\item\relax\sphinxstyleindexentry{pywfe.core.forced\_problem}\sphinxstyleindexpageref{core:\detokenize{module-pywfe.core.forced_problem}}
\item\relax\sphinxstyleindexentry{pywfe.core.model\_setup}\sphinxstyleindexpageref{core:\detokenize{module-pywfe.core.model_setup}}
\item\relax\sphinxstyleindexentry{pywfe.utils.comsol\_loader}\sphinxstyleindexpageref{utils:\detokenize{module-pywfe.utils.comsol_loader}}
\item\relax\sphinxstyleindexentry{pywfe.utils.frequency\_sweep}\sphinxstyleindexpageref{utils:\detokenize{module-pywfe.utils.frequency_sweep}}
\item\relax\sphinxstyleindexentry{pywfe.utils.io\_utils}\sphinxstyleindexpageref{utils:\detokenize{module-pywfe.utils.io_utils}}
\item\relax\sphinxstyleindexentry{pywfe.utils.modal\_assurance}\sphinxstyleindexpageref{utils:\detokenize{module-pywfe.utils.modal_assurance}}
\item\relax\sphinxstyleindexentry{pywfe.utils.shaker}\sphinxstyleindexpageref{utils:\detokenize{module-pywfe.utils.shaker}}
\end{sphinxtheindex}

\renewcommand{\indexname}{Index}
\printindex
\end{document}